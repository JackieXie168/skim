%%  Instructions for building Michael McCracken's BibDesk from source.
%%  Adam Maxwell <amaxwell@wsu.edu> is responsible for the initial revision of this document.
%%  10 March 2003  1st draft
%%  Revised 13 July 2003
%%  Panther--specific section added 1 November 2003
%%

\documentclass[11pt]{article}

\textwidth = 6.5 in
\textheight = 9 in
\oddsidemargin = 0.0 in
\evensidemargin = 0.0 in
\topmargin = 0.0 in
\headheight = 0.0 in
\headsep = 0.0 in
\parskip = 0.2in
\parindent = 0.0in

\begin{document}

\section{File Setup}
\begin{enumerate}
\item Follow the directions for CVS at http://www.sf.net/projects/bibdesk/ to pull the source tree.
\item Open the BibDesk.pbproj file in Project Builder, and click on the ``Files'' tab.
\item Open the Frameworks tree by clicking the triangle.  You should see three frameworks listed at
the bottom:
  \begin{enumerate}
  \item OmniAppKit.framework
  \item OmniBase.framework
  \item OmniFoundation.framework
  \end{enumerate}
\item Some of these may be listed in red.  If they are, you will have to compile the frameworks on
your system.  They can be downloaded from http://www.omnigroup.com/ (look under Developer/Source
Code) and need to be compiled.  If you are going to do this, the following should help:
  \begin{enumerate}
  \item Compile OmniBase, OmniFoundation, and OmniAppKit \textbf{in that order}, since
they depend on each other.
  \item The best way to do this is to set Project Builder's preferences so that build products are in
a separate directory; I use /Users/username/BuildProducts, and the intermediate and final products
go in there.  This keeps things separate from the source tree.
  \item Open up OmniBase.pbproj in Project Builder, click the Targets tab and select a Deployment
build.  To setup the framework so it will be properly embedded, go to the Settings--Simple
View--Installation Settings, select Path and enter
\begin{verbatim}@executable_path/../Frameworks\end{verbatim}  Next, go to Settings--Expert View,
click the plus--sign button, type \begin{verbatim}SKIP_INSTALL\end{verbatim} press tab, and type
\begin{verbatim}YES\end{verbatim}
For further information, see Apple's developer documentation.
  \item Click the hammer, and when finished, you will have OmniBase.framework in your build
products directory. 
  \item Open up the OmniFoundation.pbproj, click the Files tab, and find the OmniBase.framework
under ``External Frameworks and Libraries;'' if it shows up as red, get info on that framework
in Project Builder (right--click or control--click), and locate the framework using a Build
Product Relative style.  Repeat the previous setup steps, then repeat this process for the
OmniAppKit.
  \item Don't put the frameworks in /Library/Frameworks or ~/Library/Frameworks,
since you want your applications to use the version of the Omni frameworks that is installed in
the application wrapper.
  \item Hopefully you have all the Omni frameworks compiled and installed at this point.  This
discussion on http://cocoadev.com/index.pl?IncludingFrameworksInApplications is helpful, if you want
something beyond the quick--and--dirty steps I've given.
  \end{enumerate}
\item At this point, we need to satisfy BibDesk's dependencies on the Omni frameworks, just like the
Omni frameworks depend on each other.  Get info on each of the three Omni frameworks in the
BibDesk.pbproj project (right--click on their icons) and set the paths to those frameworks correctly
for your system.  Their names should change from red to black in color.  I think you probably want
them all to be the same version, so make sure the paths point to the frameworks you just installed.
\end{enumerate}

\section{Target Setup}
\begin{enumerate}
\item You can set various options in the Targets tab, e.g. Build Styles (Development vs. Deployment),
and Optimization (under GCC Compiler Settings).
\item Uncheck the test files for the BibDesk build.
\item Build and run it.
\item If there are errors, the log output is fairly descriptive.  If ProjectBuilder can't find a
file, it will list it in red, but you might have to start expanding the file tree or use the
search.  It is helpful to look at the Build Phases for the BibDesk target, too, since that will
usually give you a clue where things stopped.
\item The Omni frameworks should be copied into BibDesk.app/Contents/Frameworks automatically in the
build process, so the resulting app can run on a machine that does not have them.  You can check
this by `otool -L BibDesk.app/Contents/MacOS/BibDesk` which should reference the Omni frameworks
like this:  \begin{verbatim}
  @executable_path/../Frameworks/OmniAppKit.framework/Versions/2001A/OmniA
  ppKit (compatibility version 1.0.0, current version 1.0.0)
\end{verbatim}  This tip is from the cocoadev article.
\end{enumerate}

\section{Panther--specific notes}
\begin{enumerate}
\item To rebuild the Omni frameworks (August 2003 release), you need to double--click the
target for the framework and uncheck ``Treat all warnings as errors'' under GCC Compiler
Settings.  Since this is also set explicitly in Expert View, go to Expert View and delete
\tt{-Werror} \rm from the \tt WARNING\_CFLAGS \rm option.
\item You must set up a native target for BibDesk in order to use the new ``Fix'' debugging
option.  To do this, select the BibDesk target and choose ``Upgrade Target to Native'' from
the Project menu in XCode.
\item You may have to add the Omni frameworks to the ``Copy Files'' build phase in your
target.  To do this, drag them from the Groups \& Files inspector into the first ``Copy
Files'' phase, then get info on ``Copy Files'' to make sure that the destination is set to
``Frameworks.''
\end{enumerate}


 \end{document}
 \end
