\documentclass{article}
%\input{file}
% \include{another}
\begin{document}

%latex2rtf: This comment will only show up in the RTF conversion of a tex file.

%latex2rtf: 
%latex2rtf: 

\section{This is the first section in file \texttt{include.tex}}
\label{section01}

Here is some text to create a paragraph or two so that we
can see if this works or not.  It will be interesting to see
if the labels work properly.  As I create these testing files
I realize that I need to add parsing support for \verb#\input#
and \verb#\include# in the getSection function.   Furthermore,
after reading the \LaTeX{} book, I see that \verb#\include# 
files will all start on a new page.  This is not the case for
\verb#\input# files.  Here is a reference to the next section \ref{section02}
in this file,
and one to the first subsection in the first include file \texttt{include1} \ref{section11}. 
This is a reference \ref{section31} to the first subsubsection in 
the file \texttt{include3} that is included by \texttt{include2}.

\subsection{This is the first subsection from file \texttt{include1.tex}}
\label{section11}

Here is some text to create a paragraph or two so that we
can see if this works or not.  It will be interesting to see
if the labels work properly.  As I create these testing files
I realize that I need to add parsing support for \verb#\input#
and \verb#\include# in the getSection function.   Furthermore,
after reading the \LaTeX{} book, I see that \verb#\include# 
files will all start on a new page.  This is not the case for
\verb#\input# files.  Here is a reference to the next subsection \ref{section12}.

\subsection{This is the second subsection from file \texttt{include1.tex}}
\label{section12}

Here is some text to create a paragraph or two so that we
can see if this works or not.  It will be interesting to see
if the labels work properly.  As I create these testing files
I realize that I need to add parsing support for \verb#\input#
and \verb#\include# in the getSection function.   Furthermore,
after reading the \LaTeX{} book, I see that \verb#\include# 
files will all start on a new page.  This is not the case for
\verb#\input# files.  Here is a reference to the previous subsection \ref{section11}.

\endinput

Here is text that should never be seen.
\subsection{This is the first subsection in file \texttt{include2.tex}}
\label{section21}

Here is some text to create a paragraph or two so that we
can see if this works or not.  It will be interesting to see
if the labels work properly.  As I create these testing files
I realize that I need to add parsing support for \verb#\input#
and \verb#\include# in the getSection function.   Furthermore,
after reading the \LaTeX{} book, I see that \verb#\include# 
files will all start on a new page.  This is not the case for
\verb#\input# files.  Here is a reference to the next subsection \ref{section22}.

\subsection{This is the second subsection in file \texttt{include2.tex}}
\label{section22}

Here is some text to create a paragraph or two so that we
can see if this works or not.  It will be interesting to see
if the labels work properly.  As I create these testing files
I realize that I need to add parsing support for \verb#\input#
and \verb#\include# in the getSection function.   Furthermore,
after reading the \LaTeX{} book, I see that \verb#\include# 
files will all start on a new page.  This is not the case for
\verb#\input# files.  Here is a reference to the previous subsection \ref{section21}.
Here is a reference to the first section in \texttt{include1.tex}
\ref{section11}.

Now we include another file
\subsubsection{This is the first subsubsection from file \texttt{include3.tex}}
\label{section31}

Here is some text to create a paragraph or two so that we
can see if this works or not.  It will be interesting to see
if the labels work properly.  As I create these testing files
I realize that I need to add parsing support for \verb#\input#
and \verb#\include# in the getSection function.   Furthermore,
after reading the \LaTeX{} book, I see that \verb#\include# 
files will all start on a new page.  This is not the case for
\verb#\input# files.  Here is a reference to the next subsubsection \ref{section32}.

\subsubsection{This is the second subsubsection from file \texttt{include3.tex}}
\label{section32}

Here is some text to create a paragraph or two so that we
can see if this works or not.  It will be interesting to see
if the labels work properly.  As I create these testing files
I realize that I need to add parsing support for \verb#\input#
and \verb#\include# in the getSection function.   Furthermore,
after reading the \LaTeX{} book, I see that \verb#\include# 
files will all start on a new page.  This is not the case for
\verb#\input# files.  Here is a reference to the previous subsubsection \ref{section31}.
Here is a reference to the first subsection from \texttt{include1.tex}
\ref{section11}.

post include file.

end of include 2


% \subsection{This is the first subsection from file \texttt{include1.tex}}
\label{section11}

Here is some text to create a paragraph or two so that we
can see if this works or not.  It will be interesting to see
if the labels work properly.  As I create these testing files
I realize that I need to add parsing support for \verb#\input#
and \verb#\include# in the getSection function.   Furthermore,
after reading the \LaTeX{} book, I see that \verb#\include# 
files will all start on a new page.  This is not the case for
\verb#\input# files.  Here is a reference to the next subsection \ref{section12}.

\subsection{This is the second subsection from file \texttt{include1.tex}}
\label{section12}

Here is some text to create a paragraph or two so that we
can see if this works or not.  It will be interesting to see
if the labels work properly.  As I create these testing files
I realize that I need to add parsing support for \verb#\input#
and \verb#\include# in the getSection function.   Furthermore,
after reading the \LaTeX{} book, I see that \verb#\include# 
files will all start on a new page.  This is not the case for
\verb#\input# files.  Here is a reference to the previous subsection \ref{section11}.

\endinput

Here is text that should never be seen. should not be included!

\section{This is the second section in file \texttt{include.tex}}
\label{section02}

Here is some text to create a paragraph or two so that we
can see if this works or not.  It will be interesting to see
if the labels work properly.  As I create these testing files
I realize that I need to add parsing support for \verb#\input#
and \verb#\include# in the getSection function.   Furthermore,
after reading the \LaTeX{} book, I see that \verb#\include# 
files will all start on a new page.  This is not the case for
\verb#\input# files.  Here is a reference to the first subsection 
in \texttt{include2.tex} \ref{section21}.
Here is a reference to the first subsection in \texttt{include1.tex}
\ref{section11}. And finally one to the first section of this 
file \ref{section01}.

\end{document}

XXXX
more text 

YYYY 
