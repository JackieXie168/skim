% Essential LaTeX - Jon Warbrick 02/88

% Copyright (C) Jon Warbrick and Plymouth Polytechnic 1989
% Permission is granted to reproduce the document in any way providing
% that it is distributed for free, except for any reasonable charges for
% printing, distribution, staff time, etc.  Direct commercial
% exploitation is not permitted.  Extracts may be made from this
% document providing an acknolwledgment of the original source is
% maintained.

\documentstyle[11pt,hndout]{article}

% counters used for the sample file example

\newcounter{savesection}
\newcounter{savesubsection}

% commands to do 'LaTeX Manual-like' examples

\newlength{\egwidth}\setlength{\egwidth}{0.42\textwidth}

\newenvironment{eg}%
{\begin{list}{}{\setlength{\leftmargin}{0.05\textwidth}%
\setlength{\rightmargin}{\leftmargin}}\item[]\footnotesize}%
{\end{list}}

\newenvironment{egbox}%
{\begin{minipage}[t]{\egwidth}}%
{\end{minipage}}

\newcommand{\egstart}{\begin{eg}\begin{egbox}}
\newcommand{\egmid}{\end{egbox}\hfill\begin{egbox}}
\newcommand{\egend}{\end{egbox}\end{eg}}

% one or two other commands

\newcommand{\fn}[1]{\hbox{\tt #1}}
\newcommand{\llo}[1]{(see line #1)}
\newcommand{\lls}[1]{(see lines #1)}
\newcommand{\bs}{$\backslash$}

\title{Essential \LaTeX}
\author{Jon Warbrick}
\docnumber{U05.04--202}

\begin{document}

\maketitle

\section{Introduction}

This document is an attempt to give you all the essential
information that you will need in order to use the \LaTeX{} Document
Preparation System.  Only very basic features are covered, and a
vast amount of detail has been omitted.  In a document of this size
it is not possible to include everything that you might need to know,
and if you intend to make extensive use of the program you should
refer to a more complete reference.  Attempting to produce complex
documents using only the information found below will require
much more work than it should, and will probably produce a less
than satisfactory result.

The main reference for \LaTeX{} is {\em The \LaTeX{} User's guide and
Reference Manual\/} by Leslie Lamport (M05.04--200).  This contains
all the information that you will ever need to know about the program,
and you will need access to a copy if you are to use \LaTeX{}
seriously.

Both the Manual and this document avoid mentioning anything that
depends on the particular computer system that you are using.  This is
because \LaTeX{} is available on a number of systems and they all
differ in one way or another.  Instead, they both refer to a {\em local
guide\/} for their particular system.  For \LaTeX{} on the Prime, the
{\em local guide\/} is the handout {\em The Local Guide to \LaTeX{} on
the Plymouth Prime System\/} (U05.04--201).

\section{How does \LaTeX{} work?}

In order to use \LaTeX{} you generate a file containing
both the text that you wish to print and instructions to tell \LaTeX{}
how you want it to appear.  You will normally create
this file using your system's text editor.  You can give the file any name you
like, but it should end ``\fn{.TEX}'' to identify the file's contents.
You then get \LaTeX{} to process the file, and it creates a
new file of typesetting commands; this has the same name as your file but
the ``\fn{.TEX}'' ending is replaced by ``\fn{.DVI}''.  This stands for
`{\it D\/}e{\it v\/}ice {\it I\/}ndependent' and, as the name implies, this file
can be used to create output on a range of printing devices.
Your {\em local guide\/} will go into more detail.

Rather than encourage you to dictate exactly how your document
should be laid out, \LaTeX{} instructions allow you describe its
{\em logical structure\/}.  For example, you can think of a quotation
embedded within your text as an element of this logical structure: you would
normally expect a quotation to be displayed in a recognisable style to set it
off from the rest of the text.
A human typesetter would recognise the quotation and handle
it accordingly, but since \LaTeX{} is only a computer program it requires
your help.  There are therefore \LaTeX{} commands that allow you to
identify quotations and as a result allow \LaTeX{} to typeset them correctly.

Fundamental to \LaTeX{} is the idea of a {\em document style\/} that
determines exactly how a document will be formatted.  \LaTeX{} provides
standard document styles that describe how standard logical structures
(such as quotations) should be formatted.  You may have to supplement
these styles by specifying the formatting of logical structures
peculiar to your document, such as mathematical formulae.  You can
also modify the standard document styles or even create an entirely
new one, though you should know the basic principles of typographical
design before creating a radically new style.

There are a number of good reasons for concentrating on the logical
structure rather than on the appearance of a document.  It prevents
you from making elementary typographical errors in the mistaken
idea that they improve the aesthetics of a document---you should
remember that the primary function of document design is to make
documents easier to read, not prettier.  It is more flexible, since
you only need to alter the definition of the quotation style
to change the appearance of all the quotations in a document.  Most
important of all, logical design encourages better writing.
A visual system makes it easier to create visual effects rather than
a coherent structure; logical design encourages you to concentrate on
your writing and makes it harder to use formatting as a substitute
for good writing.

\section{A Sample \LaTeX{} file}

\begin{figure} %---------------------------------------------------------------
{\footnotesize\begin{verbatim}
 1: % SMALL.TEX -- Released 5 July 1985
 2: % USE THIS FILE AS A MODEL FOR MAKING YOUR OWN LaTeX INPUT FILE.
 3: % EVERYTHING TO THE RIGHT OF A  %  IS A REMARK TO YOU AND IS IGNORED
 4: % BY LaTeX.
 5: %
 6: % WARNING!  DO NOT TYPE ANY OF THE FOLLOWING 10 CHARACTERS EXCEPT AS
 7: % DIRECTED:        &   $   #   %   _   {   }   ^   ~   \
 8:
 9: \documentstyle[11pt,a4]{article}  % YOUR INPUT FILE MUST CONTAIN THESE
10: \begin{document}                  % TWO LINES PLUS THE \end COMMAND AT
11:                                   % THE END
12:
13: \section{Simple Text}          % THIS COMMAND MAKES A SECTION TITLE.
14:
15: Words are separated by one or    more      spaces.  Paragraphs are
16:     separated by one or more blank lines.  The output is not affected
17: by adding extra spaces or extra blank lines to the input file.
18:
19:
20: Double quotes are typed like this: ``quoted text''.
21: Single quotes are typed like this: `single-quoted text'.
22:
23: Long dashes are typed as three dash characters---like this.
24:
25: Italic text is typed like this: {\em this is italic text}.
26: Bold   text is typed like this: {\bf this is  bold  text}.
27:
28: \subsection{A Warning or Two}        % THIS MAKES A SUBSECTION TITLE.
29:
30: If you get too much space after a mid-sentence period---abbreviations
31: like etc.\ are the common culprits)---then type a backslash followed by
32: a space after the period, as in this sentence.
33:
34: Remember, don't type the 10 special characters (such as dollar sign and
35: backslash) except as directed!  The following seven are printed by
36: typing a backslash in front of them:  \$  \&  \#  \%  \_  \{  and  \}.
37: The manual tells how to make other symbols.
38:
39: \end{document}                    % THE INPUT FILE ENDS LIKE THIS
\end{verbatim}  }

\caption{A Sample \LaTeX{} File}\label{fig:sample}

\end{figure} %-----------------------------------------------------------------

\begin{figure} %---------------------------------------------------------------

% This figure will only work if the main style of this document
% is article, or something like article, because that's what the example
% file thinks it is.  Unfortunately, I can't see how to include a
% \documentstyle command in a figure!

\setcounter{savesection}{\value{section}}
\setcounter{section}{0}
\setcounter{savesubsection}{\value{subsection}}
\setcounter{subsection}{0}

\setlength{\parindent}{17pt}
\setlength{\parskip}{0pt}

\noindent\rule{\textwidth}{0.8pt}

% SMALL.TEX -- Released 5 July 1985
% USE THIS FILE AS A MODEL FOR MAKING YOUR OWN LaTeX INPUT FILE.
% EVERYTHING TO THE RIGHT OF A  %  IS A REMARK TO YOU AND IS IGNORED
% BY LaTeX.
%
% WARNING!  DO NOT TYPE ANY OF THE FOLLOWING 10 CHARACTERS EXCEPT AS
% DIRECTED:        &   $   #   %   _   {   }   ^   ~   \

% .........................................................................
% \documentstyle[a4]{article}    % YOUR INPUT FILE MUST CONTAIN THESE
% \begin{document}               % TWO LINES PLUS THE \end COMMAND AT
%                                % THE END
% .........................................................................

\section{Simple Text}          % THIS COMMAND MAKES A SECTION TITLE.

Words are separated by one or    more      spaces.  Paragraphs are
    separated by one or more blank lines.  The output is not affected
by adding extra spaces or extra blank lines to the input file.


Double quotes are typed like this: ``quoted text''.
Single quotes are typed like this: `single-quoted text'.

Long dashes are typed as three dash characters---like this.

Italic text is typed like this: {\em this is italic text}.
Bold   text is typed like this: {\bf this is  bold  text}.

\subsection{A Warning or Two}        % THIS MAKES A SUBSECTION TITLE.

If you get too much space after a mid-sentence period---abbreviations
like etc.\ are the common culprits)---then type a backslash followed by
a space after the period, as in this sentence.

Remember, don't type the 10 special characters (such as dollar sign and
backslash) except as directed!  The following seven are printed by
typing a backslash in front of them:  \$  \&  \#  \%  \_  \{  and  \}.
The manual tells how to make other symbols.

% ................
% \end{document}
% ................

\noindent\rule{\textwidth}{0.8pt}

\setcounter{section}{\value{savesection}}
\setcounter{subsection}{\value{savesubsection}}

\caption{The result of processing the sample file}\label{fig:result}

\end{figure} % ----------------------------------------------------------------

Have a look at the example \LaTeX{} file in Figure~\ref{fig:sample}.  It
is a slightly modified copy of the standard \LaTeX{} example file
\fn{SMALL.TEX}.  Your local guide will tell you how you can make
your own copy of this file.  The line numbers down the left-hand side
are not part of the file, but have been added to make it easier to
identify various portions. Also have a look at Figure~\ref{fig:result} which
shows, more or less, the result of processing this file.

\subsection{Running Text}

Most documents consist almost entirely of running text---words formed
into sentences, which are in turn formed into paragraphs---and the example file
is no exception. Describing running text poses no problems, you just type
it in naturally. In the output that it produces, \LaTeX{} will fill
lines and adjust the
spacing between words to give tidy left and right margins.
The spacing and distribution of the words in your input
file will have no effect at all on the eventual output.
Any number of spaces in your input file
are treated as a single space by \LaTeX{}, it also regards the
end of each line as a space between words \lls{15--17}.
A new paragraph is
indicated by a blank line in your input file, so don't leave
any blank lines unless you really wish to start a paragraph.

\LaTeX{} reserves a number of the less common keyboard characters for its
own use. The ten characters
\begin{quote}\begin{verbatim}
#  $  %  &  ~  _  ^  \  {  }
\end{verbatim}\end{quote}
should not appear as part of your text, because if they do
\LaTeX{} will get confused.

\subsection{\LaTeX{} Commands}

There are a number of words in the file that start `\verb|\|' \lls{9, 10 and
13}.  These are \LaTeX{} {\em commands\/} and they describe the structure
of your document.
There are a number of things that you should realise about these commands:
\begin{itemize}

\item All \LaTeX{} commands consist of a `\verb|\|' followed by one or more
characters.

\item \LaTeX{} commands should be typed using the correct mixture of upper- and
lower-case letters.  \verb|\BEGIN| is {\em not\/} the same as \verb|\begin|.

\item Some commands are placed within your text.  These are used to
switch things, like different typestyles, on and off. The \verb|\em| command
is used like this to emphasise text, normally by changing to an {\it italic\/}
typestyle \llo{25}.  The command and the text are always enclosed between
`\verb|{|' and `\verb|}|'---the `\verb|{\em|' turns the effect on and and the
`\verb|}|' turns it off.

\item There are other commands that look like
\begin{quote}\begin{verbatim}
\command{text}
\end{verbatim}\end{quote}
In this case the text is called the ``argument'' of the command.  The
\verb|\section| command is like this \llo{13}.
Sometimes you have to use curly brackets `\verb|{}|' to enclose the argument,
sometimes square brackets `\verb|[]|', and sometimes both at once.
There is method behind this apparent madness, but for the
time being you should be sure to copy the commands exactly as given.

\item When a command's name is made up entirely of letters, you must make sure
that the end of the command is marked by something that isn't a letter.
This is usually either the opening bracket around the command's argument, or
it's a space.  When it's a space, that space is always ignored by \LaTeX. We
will see later that this can sometimes be a problem.

\end{itemize}

\subsection{Overall structure}

There are some \LaTeX{} commands that must appear in every document.
The actual text of the document always starts with a
\verb|\begin{document}| command and ends with an \verb|\end{document}|
command \lls{10 and 39}.  Anything that comes after the
\verb|\end{document}| command is ignored.  Everything that comes
before the \verb|\begin{document}| command is called the
{\em preamble\/}. The preamble can only contain \LaTeX{} commands
to describe the document's style.

One command that must appear in the preamble is the
\verb|\documentstyle| command \llo{9}.  This command specifies
the overall style for the document.  Our example file is a simple
technical document, and uses the {\tt article\/} style, modified to
print in eleven-point type on A4 paper.
There are other styles that you can use, as you will
find out later on in this document.

\subsection{Other Things to Look At}

\LaTeX{} can print both opening and closing quote characters, and can manage
either of these either single or double.  To do this it uses the two quote
characters from your keyboard: {\tt `} and {\tt '}. You will probably think of
{\tt '} as the ordinary single quote character which probably looks like
{\tt\symbol{'23}} or {\tt\symbol{'15}} on your keyboard,
and {\tt `} as a ``funny''
character that probably appears as {\tt\symbol{'22}}.
You type these characters once for single quote \llo{21},  and twice for
double quotes \llo{20}. The double quote character {\tt "} itself
is almost never used.

\LaTeX{} can produce three different kinds of dashes.
A long dash, for use as a punctuation symbol, as is typed as three dash
characters in a row, like this `\verb|---|' \llo{23}.  A shorter dash,
used between numbers as in `10--20', is typed as two dash
characters in a row, while a single dash character is used as a hyphen.

From time to time you will need to include one or more of the \LaTeX{}
special symbols in your text.  Seven of them can be printed by
making them into commands by proceeding them by backslash
\llo{36}.  The remaining three symbols can be produced by more
advanced commands, as can symbols that do not appear on your keyboard
such as \dag, \ddag, \S, \pounds, \copyright, $\sharp$ and $\clubsuit$.

It is sometimes useful to include comments in a \LaTeX{} file, to remind
you of what you have done or why you did it.  Everything to the
right of a \verb|%| sign is ignored by \LaTeX{}, and so it can
be used to introduce a comment.

\section{Document Styles and Style Options}\label{sec:styles}

There are four standard document styles available in \LaTeX:
\nobreak
\begin{description}

\item[{\tt article}]  intended for short documents and articles for publication.
Articles do not have chapters, and when \verb|\maketitle| is used to generate
a title (see Section~\ref{sec:title}) it appears at the top of the first page
rather than on a page of its own.

\item[{\tt report}] intended for longer technical documents.
It is similar to
{\tt article}, except that it contains chapters and the title appears on a page
of its own.

\item[{\tt book}] intended as a basis for book publication.  Page layout is
adjusted assuming that the output will eventually be used to print on
both sides of the paper.

\item[{\tt letter}]  intended for producing personal letters.  This style
will allow you to produce all the elements of a well laid out letter:
addresses, date, signature, etc.
\end{description}

These standard styles can be modified by a number of {\em style options\/}.
They appear in square brackets after the \verb|\documentstyle| command.
Only one style can ever be used but you can have more than one style option,
in which case their names should be separated by commas.  The standard style
options are:
\begin{description}

\item[{\tt 11pt}]  prints the document using eleven-point type for the running
 text
rather that the ten-point type normally used. Eleven-point type is about
ten percent larger than ten-point.

\item[{\tt 12pt}]  prints the document using twelve-point type for the running
 text
rather than the ten-point type normally used. Twelve-point type is about
twenty percent larger than ten-point.

\item[{\tt twoside}]  causes documents in the article or report styles to be
formatted for printing on both sides of the paper.  This is the default for the
book style.

\item[{\tt twocolumn}] produces two column on each page.

\item[{\tt titlepage}]  causes the \verb|\maketitle| command to generate a
title on a separate page for documents in the \fn{article} style.
A separate page is always used in both the \fn{report} and \fn{book} styles.

\end{description}
There is one further option which, while not standard to \LaTeX{}, is very
useful in all European countries. The \fn{a4} option causes the output in all
of the standard styles to be adjusted to fit correctly on A4 paper.  \LaTeX{}
was designed in America where the standard paper is shorter and slightly wider
than A4; without this option you will find that your output looks a little
strange.

\section{Environments}

We mentioned earlier the idea of identifying a quotation to \LaTeX{} so that
it could arrange to typeset it correctly. To do this you enclose the
quotation between the commands \verb|\begin{quotation}| and
\verb|\end{quotation}|.
This is an example of a \LaTeX{} construction called an {\em environment\/}.
A number of
special effects are obtained by putting text into particular environments.

\subsection{Quotations}

There are two environments for quotations: \fn{quote} and \fn{quotation}.
\fn{quote} is used either for a short quotation or for a sequence of
short quotations separated by blank lines:

\egstart
\begin{verbatim}
US presidents ... pithy remarks:
\begin{quote}
The buck stops here.

I am not a crook.
\end{quote}
\end{verbatim}
\egmid%
US presidents have been known for their pithy remarks:
\begin{quote}
The buck stops here.

I am not a crook.
\end{quote}
\egend

Use the \fn{quotation} environment for quotations that consist of more
than one paragraph.  Paragraphs in the input are separated by blank
lines as usual:
\egstart
\begin{verbatim}
Here is some advice to remember:
\begin{quotation}
Environments for making
...other things as well.

Many problems
...environments.
\end{quotation}
\end{verbatim}
\egmid%
Here is some advice to remember:
\begin{quotation}
Environments for making quotations
can be used for other things as well.

Many problems can be solved by
novel applications of existing
environments.
\end{quotation}
\egend

\subsection{Centering and Flushing}

Text can be centred on the page by putting it within the \fn{center}
environment, and it will appear flush against the left or right margins if it
is placed within the \fn{flushleft} or \fn{flushright} environments.
Notice the
spelling of \fn{center}---unfortunately \LaTeX{} doesn't understand the English
spelling.

Text within these environments will be formatted in the normal way, in
{\samepage
particular the ends of the lines that you type are just regarded as spaces.  To
indicate a ``newline'' you need to type the \verb|\\| command.  For example:
\egstart
\begin{verbatim}
\begin{center}
one
two
three \\
four \\
five
\end{center}
\end{verbatim}
\egmid%
\begin{center}
one
two
three \\
four \\
five
\end{center}
\egend
}

\subsection{Lists}

There are three environments for constructing lists.  In each one each new
item is begun with an \verb|\item| command.  In the \fn{itemize} environment
the start of each item is given a marker, in the \fn{enumerate}
environment each item is marked by a number.  These environments can be nested
within each other in which case the amount of indentation used
is adjusted accordingly:
\egstart
\begin{verbatim}
\begin{itemize}
\item Itemized lists are handy.
\item However, don't forget
  \begin{enumerate}
  \item The `item' command.
  \item The `end' command.
  \end{enumerate}
\end{itemize}
\end{verbatim}
\egmid%
\begin{itemize}
\item Itemized lists are handy.
\item However, don't forget
  \begin{enumerate}
  \item The `item' command.
  \item The `end' command.
  \end{enumerate}
\end{itemize}
\egend

The third list making environment is \fn{description}.  In a description you
specify the item labels inside square brackets after the \verb|\item| command.
For example:
\egstart
\begin{verbatim}
Three animals that you should
know about are:
\begin{description}
  \item[gnat] A small animal...
  \item[gnu] A large animal...
  \item[armadillo] A ...
\end{description}
\end{verbatim}
\egmid%
Three animals that you should
know about are:
\begin{description}
  \item[gnat] A small animal that causes no end of trouble.
  \item[gnu] A large animal that causes no end of trouble.
  \item[armadillo] A medium-sized animal.
\end{description}
\egend

\subsection{Tables}

Because \LaTeX{} will almost always convert a sequence of spaces
into a single space,
it can be rather difficult to lay out tables.  See what happens in this example
\nolinebreak
\egstart
\begin{verbatim}
\begin{flushleft}
Income  Expenditure Result   \\
20s 0d  19s 11d     happiness \\
20s 0d  20s 1d      misery  \\
\end{flushleft}
\end{verbatim}
\egmid%
\begin{flushleft}
Income  Expenditure Result   \\
20s 0d  19s 11d     happiness \\
20s 0d  20s 1d      misery  \\
\end{flushleft}
\egend

The \fn{tabbing} environment overcomes this problem. Within it you set
tabstops and tab to them much like you do on a typewriter.  Tabstops are
set with the \verb|\=| command, and the \verb|\>| command moves to the
next stop.  The
\verb|\\| command is used to separate each line.  A line that ends \verb|\kill|
produces no output, and can be used to set tabstops:
\nolinebreak
\egstart
\begin{verbatim}
\begin{tabbing}
Income \=Expenditure \=    \kill
Income \>Expenditure \>Result \\
20s 0d \>19s 11d \>Happiness   \\
20s 0d \>20s 1d  \>Misery    \\
\end{tabbing}
\end{verbatim}
\egmid%
\begin{tabbing}
Income \=Expenditure \=    \kill
Income \>Expenditure \>Result \\
20s 0d \>19s 11d \>Happiness   \\
20s 0d \>20s 1d  \>Misery    \\
\end{tabbing}
\egend

Unlike a typewriter's tab key, the \verb|\>| command always moves to the next
tabstop in sequence, even if this means moving to the left.  This can cause
text to be overwritten if the gap between two tabstops is too small.

\subsection{Verbatim Output}

Sometimes you will want to include text exactly as it appears on a terminal
screen.  For example, you might want to include part of a computer program.
Not only do you want \LaTeX{} to stop playing around with the layout of your
text, you also want to be able to type all the characters on your keyboard
without confusing \LaTeX. The \fn{verbatim} environment has this effect:
\egstart
\begin{flushleft}
\verb|The section of program in|  \\
\verb|question is:|               \\
\verb|\begin{verbatim}|           \\
\verb|{ this finds %a & %b }|     \\[2ex]

\verb|for i := 1 to 27 do|        \\
\ \ \ \verb|begin|                \\
\ \ \ \verb|table[i] := fn(i);|   \\
\ \ \ \verb|process(i)|           \\
\ \ \ \verb|end;|                 \\
\verb|\end{verbatim}|
\end{flushleft}
\egmid%
The section of program in
question is:
\begin{verbatim}
{ this finds %a & %b }

for i := 1 to 27 do
   begin
   table[i] := fn(i);
   process(i)
   end;

\end{verbatim}
\egend

\section{Type Styles}

We have already come across the \verb|\em| command for changing
typeface.  Here is a full list of the available typefaces:
\begin{quote}\begin{tabbing}
\verb|\sc|~~ \= \sc Small Caps~~~ \= \verb|\sc|~~ \= \sc Small Caps~~~
                                  \= \verb|\sc|~~ \=                   \kill
\verb|\rm|   \> \rm Roman         \> \verb|\it|   \> \it Italic
                                  \> \verb|\sc|   \> \sc Small Caps    \\
\verb|\em|   \> \em Emphatic      \> \verb|\sl| \> \sl Slanted
                                  \> \verb|\tt|   \> \tt Typewriter     \\
\verb|\bf|   \> \bf Boldface      \> \verb|\sf| \> \sf Sans Serif
\end{tabbing}\end{quote}

Remember that these commands are used {\em inside\/} a pair of braces to limit
the amount of text that they effect.  In addition to the eight typeface
commands, there are a set of commands that alter the size of the type.  These
commands are:
\begin{quotation}\begin{tabbing}
\verb|\footnotesize|~~ \= \verb|\footnotesize|~~ \= \verb|\footnotesize| \=
 \kill
\verb|\tiny|           \> \verb|\small|          \> \verb|\large|        \>
\verb|\huge|  \\
\verb|\scriptsize|     \> \verb|\normalsize|     \> \verb|\Large|        \>
\verb|\Huge|  \\
\verb|\footnotesize|   \>                        \> \verb|\LARGE|
\end{tabbing}\end{quotation}

\section{Sectioning Commands and Tables of Contents}

Technical documents, like this one, are often divided into sections.
Each section has a heading containing a title and a number for easy
reference.  \LaTeX{} has a series of commands that will allow you to identify
different sorts of sections.  Once you have done this \LaTeX{} takes on the
responsibility of laying out the title and of providing the numbers.

The commands that you can use are:
\begin{quote}\begin{tabbing}
\verb|\subsubsection| \= \verb|\subsubsection|~~~~~~~~~~ \=           \kill
\verb|\chapter|       \> \verb|\subsection|    \> \verb|\paragraph|    \\
\verb|\section|       \> \verb|\subsubsection| \> \verb|\subparagraph| \\
\end{tabbing}\end{quote}
The naming of these last two is unfortunate, since they do not really have
anything to do with `paragraphs' in the normal sense of the word; they are just
lower levels of section.  In most document styles, headings made with
\verb|\paragraph| and \verb|\subparagraph| are not numbered.  \verb|\chapter|
is not available in document style \fn{article}.  The commands should be used
in the order given, since sections are numbered within chapters, subsections
within sections, etc.

A seventh sectioning command, \verb|\part|, is also available.  Its use is
always optional, and it is used to divide a large document into series of
parts.  It does not alter the numbering used for any of the other commands.

Including the command \verb|\tableofcontents| in you document will cause a
contents list to be included, containing information collected from the various
sectioning commands.  You will notice that each time your document is run
through \LaTeX{} the table of contents is always made up of the headings from
the previous version of the document.  This is because \LaTeX{} collects
information for the table as it processes the document, and then includes it
the next time it is run.  This can sometimes mean that the document has to be
processed through \LaTeX{} twice to get a correct table of contents.

\section{Producing Special Symbols}

You can include in you \LaTeX{} document a wide range of symbols that do not
appear on you your keyboard. For a start, you can add an accent to any letter:
\begin{quote}\begin{tabbing}

\t{oo} \= \verb|\t{oo}|~~~ \=
\t{oo} \= \verb|\t{oo}|~~~ \=
\t{oo} \= \verb|\t{oo}|~~~ \=
\t{oo} \= \verb|\t{oo}|~~~ \=
\t{oo} \= \verb|\t{oo}|~~~ \=
\t{oo} \=                       \kill

\a`{o} \> \verb|\`{o}|  \> \~{o}  \> \verb|\~{o}|  \> \v{o}  \> \verb|\v{o}| \>
\c{o}  \> \verb|\c{o}|  \> \a'{o} \> \verb|\'{o}|  \\
\a={o} \> \verb|\={o}|  \> \H{o}  \> \verb|\H{o}|  \> \d{o}  \> \verb|\d{o}| \>
\^{o}  \> \verb|\^{o}|  \> \.{o}  \> \verb|\.{o}|  \\
\t{oo} \> \verb|\t{oo}| \> \b{o}  \> \verb|\b{o}|  \\  \"{o} \> \verb|\"{o}| \>
\u{o}  \> \verb|\u{o}|  \\
\end{tabbing}\end{quote}

A number of other symbols are available, and can be used by including the
following commands:
\begin{quote}\begin{tabbing}

\LaTeX~\= \verb|\copyright|~~~~ \= \LaTeX~\= \verb|\copyright|~~~~ \=
\LaTeX~\=  \kill

\dag       \> \verb|\dag|       \> \S     \> \verb|\S|     \>
\copyright \> \verb|\copyright| \\
\ddag      \> \verb|\ddag|      \> \P     \> \verb|\P|     \>
\pounds    \> \verb|\pounds|    \\
\oe        \> \verb|\oe|        \> \OE    \> \verb|\OE|    \>
\ae        \> \verb|\AE|        \\
\AE        \> \verb|\AE|        \> \aa    \> \verb|\aa|    \>
\AA        \> \verb|\AA|        \\
\o         \> \verb|\o|         \> \O     \> \verb|\O|     \>
\l         \> \verb|\l|         \\
\L         \> \verb|\L|         \> \ss    \> \verb|\ss|    \>
?`         \> \verb|?`|         \\
!`         \> \verb|!`|         \> \ldots \> \verb|\ldots| \>
\LaTeX     \> \verb|\LaTeX|     \\
\end{tabbing}\end{quote}
There is also a \verb|\today| command that prints the current date. When you
use these commands remember that \LaTeX{} will ignore any spaces that
follow them, so that you can type `\verb|\pounds 20|' to get `\pounds 20'.
However, if you type `\verb|LaTeX is wonderful|' you will get `\LaTeX is
wonderful'---notice the lack of space after \LaTeX.
To overcome this problem you can follow any of these commands by a
pair of empty brackets and then any spaces that you wish to include,
and you will see that
\verb|\LaTeX{} really is wonderful!| (\LaTeX{} really is wonderful!).

Finally, \LaTeX{} `math' mode, normally used to layout mathematical
formulae, gives access to an even larger range of symbols, including the
upper and lower case greek mathematical alphabet, calligraphic letters,
mathematical operators and relations, arrows and a whole lot more.  This
document has not mentioned math mode, and it isn't going to either, so you
will have to refer to the manual if you really need something that you
can't get otherwise.

\section{Titles}\label{sec:title}

Most documents have a title.  To title a \LaTeX{} document, you include the
following commands in your document, usually just after
\verb|begin{document}|.
\begin{quote}\footnotesize\begin{verbatim}
\title{required title}
\author{required author}
\date{required date}
\maketitle
\end{verbatim}\end{quote}
If there are several authors, then their names should be separated by
\verb|\and|; they can also be separated by \verb|\\| if you want them to be
centred on different lines.  If the \verb|\date| command is left out, then the
current date will be printed.
\egstart
\begin{verbatim}
\title{Essential \LaTeX}
\author{J Warbrick\and A N Other}
\date{14th February 1988}
\maketitle
\end{verbatim}
\egmid
\begin{center}
{\normalsize Essential \LaTeX}\\[4ex]
J Warbrick\hspace{1em}A N Other\\[2ex]
14th February 1988
\end{center}
\egend

The exact appearance of the title varies depending on
the document style.  In styles \fn{report} and \fn{book} the title appears on a
page of its own. In the \fn{article} style it normally appears at the top
of the first page, the style option \fn{titlepage} will alter this (see
Section~\ref{sec:styles}).

\section{Letters}

Producing letters is simple with \LaTeX{}. To do this you use the document
style {\tt letter}. You can make any number of
letters with a single input file.
Your name and address, which are likely to be
the same for all letters, are given once at the top of the file.  Each letter
is produced by a {\tt letter} environment, having the name and address of the
recipient as its argument.  The letter itself begins with an \verb|\opening|
command to generate the salutation.

The letter ends with a \verb|\closing| command, you can use the commands
\verb|\encl| and \verb|\cc| to generate lists of enclosures and people to whom
you are sending copies. Any text that follows the \verb|\closing| must be
proceeded by a \verb|\ps| command.  This command produces no text---you'll have
to type ``P.S.'' yourself---but is needed to format the additional text
correctly.

Perhaps an example will make this clearer:
\begin{quote}\footnotesize\begin{verbatim}
\documentstyle{letter}
\begin{document}

\address{1234 Avenue of the Armadillos \\
         Gnu York, G.Y. 56789}
\signature{R. (Ma) Dillo \\ Director of Cuisine}

\begin{letter}{G. Natheniel Picking \\
               Acme Exterminators \\
               Illinois}

\opening{Dear Nat,}

I'm afraid that the armadillo problem is still with us.
I did everything ...

... and I hope that you can get rid of the nasty
beasts this time.

\closing{Best Regards,}
\cc{Jimmy Carter\\Richard M. Nixon}
\end{letter}

\end{document}
\end{verbatim}\end{quote}

\section{Errors}

When you create a new input file for \LaTeX{} you will probably make mistakes.
Everybody does, and it's nothing to be worried about.  As with most computer
programs, there are two sorts of mistake that you can make: those that \LaTeX{}
notices and those that it doesn't.  To take a rather silly example, since
\LaTeX{} doesn't understand what you are saying it isn't going to be worried if
you mis-spell some of the words in your text.  You will just have to accurately
proof-read your printed output.  On the other hand, if you mis-spell one of
the environment names in your file then \LaTeX won't know what you want it
to do.

When this sort of thing happens, \LaTeX{} prints an error message on your
terminal screen and then stops and waits for you to take some action.
Unfortunately, the error messages that it produces are rather user-unfriendly
and not a little frightening.  Nevertheless, if you know where to look they
will probably tell you where the error is and went wrong.

Consider what would happen if you mistyped \verb|\begin{itemize}| so that it
became \verb|\begin{itemie}|.  When \LaTeX{} processes this instruction, it
displays the following on your terminal:
\begin{quote}\footnotesize\begin{verbatim}
LaTeX error.  See LaTeX manual for explanation.
              Type  H <return>  for immediate help.
! Environment itemie undefined.
\@latexerr ...for immediate help.}\errmessage {#1}
                                                  \endgroup
l.140 \begin{itemie}

?
\end{verbatim}\end{quote}
After typing the `?' \LaTeX{} stops and waits for you to tell it what to do.

The first two lines of the message just tell you that the error was detected by
\LaTeX{}. The third line, the one that starts `!' is the {\em error indicator}.
 It
tells you what the problem is, though until you have had some experience of
\LaTeX{} this may not mean a lot to you.  In this case it is just telling you
that it doesn't recognise an environment called \fn{itemie}.
The next two lines tell you what
\LaTeX{} was doing when it found the error, they are irrelevant at the moment
and can be ignored. The final line is called the {\em error locator}, and is
a copy of the line from your file that caused the problem.
It start with a line number to help you to find it in your file, and
if the error was in the middle of a line it will be shown
broken at the point where \LaTeX{} realised that there was an error.  \LaTeX{}
can sometimes pass the point where the real error is before discovering that
something is wrong, but it doesn't usually get very far.

At this point you could do several things.  If you knew enough about \LaTeX{}
you might be able to fix the problem, or you could type `X' and press the
return key to stop \LaTeX{} running while you go and correct the error.  The
best thing to do, however, is just to press the return key.  This will allow
\LaTeX{} to go on running as if nothing had happened.  If you have made one
mistake, then you have probably made several and you may as well try to find
them all in one go.  It's much more efficient to do it this way than to run
\LaTeX{} over and over again fixing one error at a time. Don't worry about
remembering what the errors were---a copy of all the error messages is being
saved in a {\em log\/} file so that you can look at them afterwards.  See your
{\em local guide\/} to find out what this file is called.

If you look at the line that caused the error it's normally obvious what the
problem was.  If you can't work out what you problem is look at the hints
below, and if they don't help consult Chapter~6 of the manual.  It contains a
list of all of the error messages that you are likely to encounter together with
some hints as to what may have caused them.

Some of the most common mistakes that cause errors are
\begin{itemize}
\item A mis-spelt command or environment name.
\item Improperly matched `\verb|{|' and `\verb|}|'---remember that they should
 always
come in pairs.
\item Trying to use one of the ten special characters \verb|# $ % & _ { } ~ ^|
and \verb|\| as an ordinary printing symbol.
\item A missing \verb|\end| command.
\item A missing command argument (that's the bit enclosed in '\verb|{|' and
`\verb|}|').
\end{itemize}

One error can get \LaTeX{} so confused that it reports a series of spurious
errors as a result.  If you have an error that you understand, followed by a
series that you don't, then try correcting the first error---the rest
may vanish as if by magic.

Sometimes \LaTeX{} may write a {\tt *} and stop without an error message.  This
is normally caused by a missing \verb|\end{document}| command, but other errors
can cause it.  If this happens type \verb|\stop| and press the return key.

Finally, \LaTeX{} will sometimes print {\em warning\/} messages.  They report
problems that were not bad enough to cause \LaTeX{} to stop processing, but
nevertheless may require investigation.  The most common problems are
`overfull' and `underfull' lines of text.  A message like:
\begin{quote}\footnotesize\begin{verbatim}
Overfull \hbox (10.58649pt too wide) in paragraph at lines 172--175
[]\tenrm Mathematical for-mu-las may be dis-played. A dis-played
\end{verbatim}\end{quote}
indicates that \LaTeX{} could not find a good place to break a line when laying
out a paragraph.  As a result, it was forced to let the line stick out into the
right-hand margin, in this case by 10.6 points.  Since a point is about 1/72nd
of an inch this may be rather hard to see, but it will be there none the less.

This particular problem happens because \LaTeX{} is rather fussy about line
breaking, and it would rather generate a line that is too long than generate a
paragraph that doesn't meet its high standards.  The simplest way around the
problem is to enclose the entire offending paragraph between
\verb|\begin{sloppypar}| and \verb|\end{sloppypar}| commands.  This tells
\LaTeX{} that you are happy for it to break its own rules while it is working on
that particular bit of text.

Alternatively, messages about ``Underfull \verb|\hbox|'es'' may appear.
These are lines that had to have more space inserted between
words than \LaTeX{} would have liked.  In general there is not much that you
can do about these.  Your output will look fine, even if the line looks a bit
stretched.  About the only thing you could do is to re-write the offending
paragraph!

\section{A Final Reminder}

You now know enough \LaTeX{} to produce a wide range of documents.  But this
document has only scratched the surface of the
things that \LaTeX{} can do.  This entire document was itself produced with
\LaTeX{} (with no sticking things in or clever use of a photocopier) and even
it hasn't used all the features that it could.  From this you may get some
feeling for the power that \LaTeX{} puts at your disposal.

Please remember what was said in the introduction: if you {\bf do} have a
complex document to produce then {\bf go and read the manual}.  You will be
wasting your time if you rely only on what you have read here.

One other warning: having
dabbled with \LaTeX{} your documents will never be the same again \ldots.

\end{document}
