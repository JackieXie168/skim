\documentclass[frenchb]{article}
\usepackage[T1]{fontenc}
\usepackage[applemac]{inputenc}

\makeatletter

\newcommand{\noun}[1]{\textsc{#1}}

\usepackage{babel}
\makeatother
\begin{document}

\title{Les donn�s chez les Chartreux}


\author{Charles de Miramon}

\maketitle
� la fin du \noun{xii}\textsuperscript{e} si�cle, des mouvements religieux 
la�cs
se d�veloppent en Occident. � c�t� des moines et des clercs �merge
une �lite de la�cs qui souhaite vivre une vie � mi-chemin entre le
statut d'un religieux et celui d'un la�c. Certains de ces mouvements
ont �t� bien �tudi�s comme les b�guines. Les donn�s ont �t� longtemps
ignor�s par l'historiographie alors qu'il s'agit d'un ph�nom�ne de
grande ampleur.

Les donn�s sont des la�cs ou, plus rarement, des clercs qui passent
contrat avec une institution religieuse : un h�pital, une l�proserie,
un monast�re. En �change de la d�volution de l'ensemble de leurs biens
et de leur personne physique, ils re�oivent la confraternit� de l'institution,
c'est-�-dire la possibilit� de participer � ses biens temporels et
spirituels. S'ils sont astreints � une vie religieuse -- port d'un
habit et surtout d'un insigne, participation aux activit�s quotidiennes
du monast�re et de l'h�pital --, ils ne prononcent pas de voeux monastiques
et ne sont pas g�n�ralement oblig�s � la chastet� ou � la pauvret�.

\end{document}
