\documentclass[a4paper,12pt]{article}

\addtolength{\evensidemargin}{-25pt}
\addtolength{\oddsidemargin}{-25pt}

\def\displayandname#1{\rlap{$\displaystyle\csname #1\endcsname$}%
                      \qquad \texttt{\char92 #1}}
\def\mathlexicon#1{$$\vcenter{\halign{\displayandname{##}\hfil&&\qquad
                   \displayandname{##}\hfil\cr #1}}$$}

\begin{document}

\title{Getting Started with \LaTeX}
\author{David R. Wilkins}
\date{2nd Edition\\[3pt]
Copyright \copyright\ David R. Wilkins 1995}
\maketitle

\tableofcontents

\section{Introduction to \LaTeX}

\subsection{What is \LaTeX?}

\LaTeX\ is a computer program for typesetting documents.  It
takes a computer file, prepared according to the rules of
\LaTeX\, and converts it to a form that may be printed on a
high-quality printer, such as a laser writer, to produce
a printed document of a quality comparable with good
quality books and journals.  Simple documents, which do
not contain mathematical formulae or tables may be produced
very easily: effectively all one has to do is to type the
text straight in (though observing certain rules relating to
quotation marks and punctuation dashes).  Typesetting
mathematics is somewhat more complicated, but even here
\LaTeX\ is comparatively straightforward to use when one
considers the complexity of some of the formulae that it
has to produce and the large number of mathematical symbols
which it has to produce.

\LaTeX\ is one of a number of `dialects' of \TeX, all based on the
version of \TeX\ created by D. E. Knuth which is known as
Plain \TeX.  \LaTeX\ (created by L. B. Lamport) is one of
these `dialects'.  It is particularly suited to the
production of long articles and books, since it has
facilities for the automatic numbering of chapters, sections,
theorems, equations etc., and also has facilities for
cross-referencing.  It is probably one of the most suitable
version of \LaTeX\ for beginners to use.

\subsection{A Typical \LaTeX\ Input File}

In order to produce a document using \LaTeX, we must first
create a suitable \emph{input file} on the computer.  We
apply the \LaTeX\ program to the input file and then use the
printer to print out the so-called `DVI' file produced by
the \LaTeX\ program (after first using another program to
translate the  `DVI' file into a form that the printer
can understand).  Here is an example of a typical
\LaTeX\ input file:
\begin{quote}
\begin{verbatim}
\documentclass[a4paper,12pt]{article}
\begin{document}

The foundations of the rigorous study of \textit{analysis}
were laid in the nineteenth century, notably by the
mathematicians Cauchy and Weierstrass. Central to the
study of this subject are the formal definitions of
\textit{limits} and \textit{continuity}.

Let $D$ be a subset of $\bf R$ and let
$f \colon D \to \textbf{R}$ be a real-valued function on
$D$. The function $f$ is said to be \textit{continuous} on
$D$ if, for all $\epsilon > 0$ and for all $x \in D$,
there exists some $\delta > 0$ (which may depend on $x$)
such that if $y \in D$ satisfies
\[ |y - x| < \delta \]
then
\[ |f(y) - f(x)| < \epsilon. \]

One may readily verify that if $f$ and $g$ are continuous
functions on $D$ then the functions $f+g$, $f-g$ and
$f.g$ are continuous. If in addition $g$ is everywhere
non-zero then $f/g$ is continuous.

\end{document}

\end{verbatim}
\end{quote}
When we apply \LaTeX\ to these paragraphs we produce the text
\begin{quotation}
\normalsize
The foundations of the rigorous study of \textit{analysis}
were laid in the nineteenth century, notably by the
mathematicians Cauchy and Weierstrass. Central to the
study of this subject are the formal definitions of
\textit{limits} and \textit{continuity}.

Let $D$ be a subset of $\bf R$ and let
$f \colon D \to \textbf{R}$ be a real-valued function on
$D$. The function $f$ is said to be \textit{continuous} on
$D$ if, for all $\epsilon > 0$ and for all $x \in D$,
there exists some $\delta > 0$ (which may depend on $x$)
such that if $y \in D$ satisfies
\[ |y - x| < \delta\] 
then
\[ |f(y) - f(x)| < \epsilon.\] 

One may readily verify that if $f$ and $g$ are continuous
functions on $D$ then the functions $f+g$, $f-g$ and
$f.g$ are continuous. If in addition $g$ is everywhere
non-zero then $f/g$ is continuous.
\end{quotation}

This example illustrates various features of \LaTeX.  Note
that the lines
\begin{quote}
\begin{verbatim}
\documentclass[a4paper,12pt]{article}
\begin{document}
\end{verbatim}
\end{quote}
are placed at the beginning of the input file.  These are followed
by the main body of the text, followed by the concluding line
\begin{quote}
\begin{verbatim}
\end{document}
\end{verbatim}
\end{quote}
Note also that, although most characters occurring in this file
have their usual meaning, yet there are special characters such
as \verb/\/, \verb/$/, \verb/{/ and \verb/}/
which have special
meanings within \LaTeX.  Note in particular that there are
sequences of characters which begin with a `backslash'
\verb/\/ which are used to produce mathematical symbols and
Greek letters and to accomplish tasks such as changing fonts.
These sequences of characters are known as
\emph{control sequences}.

\subsection{Characters and Control Sequences}

We now describe in more detail some of the features of
\LaTeX\ illustrated in the above example.

Most characters on the keyboard, such as letters and
numbers, have their usual meaning. However the characters
\begin{quote}
\begin{verbatim}
\ { } $ ^ _ % ~ # &
\end{verbatim}
\end{quote}
are used for special purposes within \LaTeX. Thus typing one of
these characters will not produce the corresponding character
in the final document. Of course these characters are very
rarely used in ordinary text, and there are methods of
producing them when they are required in the final document.

In order to typeset a mathematical document it is
necessary to produce a considerable number of special
mathematical symbols. One also needs to be able to
change fonts. Also mathematical documents often contain
arrays of numbers or symbols (matrices) and other complicated
expressions. These are produced in \LaTeX\ using \emph{control
sequences}. Most control sequences consist of a backslash
\verb/\/ followed by a string of (upper or lower case) letters.
For example, \verb/\alpha/, \verb/\textit/ and \verb/\sum/
are control sequences.

In the example above we used the control sequences
\verb/\textit/ and \verb/\textbf/ to change the font to
\textit{italic}
and \textbf{boldface} respectively.  Also we used the control
sequences \verb/\to/, \verb/\in/, \verb/\delta/ and
\verb/\epsilon/ to produce the mathematical symbols 
$\to$
and 
$\in$
and the Greek letters 
$\delta$
and 
$\epsilon$

There is another variety of control sequence which consists
of a backslash followed by a \emph{single} character that
is not a letter.  Examples of control sequences of this sort
are \verb/\{/, \verb/\"/ and \verb/\$/.

The special characters \verb/{/ and \verb/}/ are used for
\emph{grouping} purposes.  Everything enclosed within
matching pair of such brackets is treated as a single unit.
We have applied these brackets in the example above whenever
we changed fonts.  We shall see other instances where one needs
to use \verb/{/ and \verb/}/ in \LaTeX\ to group words and symbols
together (e.g., when we need to produce superscripts and
subscripts which contain more than one symbol).

The special character \verb/$/ is used when one is
changing from ordinary text to a mathematical expression
and when one is changing back to ordinary text. Thus we
used
\begin{quote}
\begin{verbatim}
for all $\epsilon > 0$ and for all $x \in D$,
\end{verbatim}
\end{quote}
to produce the phrase
\begin{quotation}
for all $\epsilon > 0$ and for all $x \in D$,
\end{quotation}
in the example given above. Note also that we used
\verb/\[/ and \verb/\]/ in the example above to mark
the beginning and end respectively of a mathematical formula that
is displayed on a separate line.

The remaining special characters
\begin{quote}
\begin{verbatim}
^ _ % ~ # &
\end{verbatim}
\end{quote}
have special purposes within \LaTeX\ that we shall discuss
later.

\section{Producing Simple Documents using \LaTeX}

\subsection{Producing a \LaTeX\ Input File}

We describe the structure of a typical \LaTeX\ input file.

The first line of the input file should consist of a
\verb/\documentclass/ command.  The recommended such
\verb/\documentclass/ command for mathematical articles
and similar documents has the form
\begin{quote}
\begin{verbatim}
\documentclass[a4paper,12pt]{article}
\end{verbatim}
\end{quote}
(You do not have to worry about what this command means when
first learning to use \LaTeX: its effect is to ensure that the
final document is correctly positioned on A4 size paper and
that the text is of a size that is easy to read.)  There are
variants of this \verb/\documentclass/ command which are
appropriate for letters or for books.

The \verb/documentstyle/ command may be followed by certain
other optional commands, such as the \verb/\pagestyle/ command.
It is not necessary to find out about these commands when first
learning to use \LaTeX.

After the \verb/\documentclass/ command and these other
optional commands, we place the command
\begin{quote}
\begin{verbatim}
\begin{document}
\end{verbatim}
\end{quote}

This command is then followed by the main body of the text,
in the format prescribed by the rules of \LaTeX.

Finally, we end the input file with a line containing the
command
\begin{quote}
\begin{verbatim}
\end{document}
\end{verbatim}
\end{quote}

\subsection{Producing Ordinary Text using \LaTeX}

To produce a simple document using \LaTeX\ one should create a
\LaTeX\ input file, beginning with a \verb/\documentclass/
command and the \verb/\begin{document}/ command, as
described above.  The input file should end with the
\verb/\end{document}/ command, and the text of the
document should be sandwiched between the
\verb/\begin{document}/ and \verb/\end{document}/
commands in the manner described below.

If one merely wishes to type in ordinary text, without
complicated mathematical formulae or special effects such
as font changes, then one merely has to type it in as it
is, leaving a completely blank line between successive
paragraphs.  You do not have to worry about paragraph
indentation: \LaTeX\ will automatically indent all paragraphs
with the exception of the first paragraph of a new section
(unless you take special action to override the conventions
adopted by \LaTeX) 

For example, suppose that we wish to create a document
containing the following paragraphs:
\begin{quotation}
\normalsize
\noindent
If one merely wishes to type in ordinary text, without
complicated mathematical formulae or special effects such
as font changes, then one merely has to type it in as it
is, leaving a completely blank line between successive
paragraphs.

You do not have to worry about paragraph indentation:
all paragraphs will be indented with the exception of
the first paragraph of a new section.
One must take care to distinguish between the `left quote'
and the `right quote' on the computer terminal.  Also, one
should use two `single quote' characters in succession if
one requires ``double quotes''.  One should never use the
(undirected) `double quote' character on the computer
terminal, since the computer is unable to tell whether it
is a `left quote' or a `right quote'.  One also has to
take care with dashes: a single dash is used for
hyphenation, whereas three dashes in succession are required
to produce a dash of the sort used for punctuation---such as
the one used in this sentence.
\end{quotation}
To create this document using \LaTeX\ we use the following
input file:
\begin{quote}
\begin{verbatim}
\documentclass[a4paper,12pt]{article}
\begin{document}

If one merely wishes to type in ordinary text, without
complicated mathematical formulae or special effects such
as font changes, then one merely has to type it in as it
is, leaving a completely blank line between successive
paragraphs.

You do not have to worry about paragraph indentation:
all paragraphs will be indented with the exception of
the first paragraph of a new section.

One must take care to distinguish between the `left quote'
and the `right quote' on the computer terminal.  Also, one
should use two `single quote' characters in succession if
one requires ``double quotes''.  One should never use the
(undirected) `double quote' character on the computer
terminal, since the computer is unable to tell whether it
is a `left quote' or a `right quote'.  One also has to
take care with dashes: a single dash is used for
hyphenation, whereas three dashes in succession are required
to produce a dash of the sort used for punctuation---such as
the one used in this sentence.

\end{document}

\end{verbatim}
\end{quote}

Having created the input file, one then has to run it
through the \LaTeX\ program and then print it out the
resulting output file (known as a `DVI' file).

\subsection{Blank Spaces and Carriage Returns in the Input File}

\LaTeX\ treats the carriage return at the end of a line
as though it were a blank space.  Similarly \LaTeX\ treats
tab characters as blank spaces.  Moreover, \LaTeX\ regards
a sequence of blank spaces as though it were a single
space, and similarly it will ignore blank spaces at the
beginning or end of a line in the input file.  Thus, for
example, if we type
\begin{quote}
\begin{verbatim}
This is
    a
        silly
  example   of   a
file with many spaces.


                   This is the beginning
of a new paragraph.
\end{verbatim}
\end{quote}
then we obtain
\begin{quotation}
This is
    a
        silly
  example   of   a
file with many spaces.


                   This is the beginning
of a new paragraph.
\end{quotation}

It follows immediately from this that one will obtain
the same results whether one types one space or two spaces
after a full stop: \LaTeX\ does not distinguish between the
two cases.

Any spaces which follow a control sequence will be ignored
by \LaTeX.

If you really need a blank space in the final document
following whatever is produced by the control sequence,
then you must precede this blank by a
\emph{backslash} \verb/\/.  Thus in order to obtain the
sentence
\begin{quotation}
\LaTeX\ is a very powerful computer typesetting program.
\end{quotation}
we must type
\begin{quote}
\begin{verbatim}
\LaTeX\ is a very powerful computer typesetting program.
\end{verbatim}
\end{quote}
(Here the control sequence \verb/TeX/ is used to produce
the \LaTeX\ logo.)

In general, preceding a blank space by a backslash
forces \LaTeX\ to include the blank space in the final
document.

As a general rule, you should never put a blank space after
a left parenthesis or before a right parenthesis.  If you were
to put a blank space in these places, then you run the risk
that \LaTeX\ might start a new line immediately after the left
parenthesis or before the right parenthesis, leaving the
parenthesis marooned at the beginning or end of a line.


\subsection{Quotation Marks and Dashes}

Single quotation marks are produced in \LaTeX\ using
\verb/`/ and \verb/'/.  Double quotation marks are
produced by typing \verb/``/ and \verb/''/.  (The
`undirected double quote character \verb/"/ produces
double right quotation marks: it should \emph{never} be
used where left quotation marks are required.)

\LaTeX\ allows you to produce dashes of various length, known as
`hyphens', `en-dashes' and `em-dashes'.  Hyphens are obtained
in \LaTeX\ by typing \verb/-/, en-dashes by typing \verb/--/ and
em-dashes by typing \verb/---/.

One normally uses en-dashes when specifying a range of numbers.
Thus for example, to specify a range of page numbers, one would type
\begin{quote}
\begin{verbatim}
on pages 155--219.
\end{verbatim}
\end{quote}

Dashes used for punctuating are often typeset as em-dashes,
especially in older books.  These are obtained by typing
\verb/---/.

The dialogue
\begin{quotation}
``You \emph{were} a little grave,'' said Alice.

``Well just then I was inventing a new way of
getting over a gate---would you like to hear it?''

``Very much indeed,'' Alice said politely.

``I'll tell you how I came to think of it,'' said
the Knight.  ``You see, I said to myself `The only
difficulty is with the feet: the \emph{head} is
high enough already.'  Now, first I put my head on
the top of the gate---then the head's high
enough---then I stand on my head---then the feet
are high enough, you see---then I'm over, you see.''
\end{quotation}
(taken from \emph{Alice through the Looking Glass}, by Lewis Carroll)
illustrates the use of quotation marks and dashes.  It is obtained in
\LaTeX\ from the following input:
\begin{quote}
\begin{verbatim}
``You \emph{were} a little grave,'' said Alice.

``Well just then I was inventing a new way of
getting over a gate---would you like to hear it?''

``Very much indeed,'' Alice said politely.

``I'll tell you how I came to think of it,'' said
the Knight.  ``You see, I said to myself `The only
difficulty is with the feet: the \emph{head} is
high enough already.'  Now, first I put my head on
the top of the gate---then the head's high
enough---then I stand on my head---then the feet
are high enough, you see---then I'm over, you see.''
\end{verbatim}
\end{quote}

Sometimes you need single quotes immediately following double quotes,
or vica versa, as in
\begin{quotation}
``I regard computer typesetting as being reasonably
`straightforward'\,'' he said.
\end{quotation}
The way to typeset this correctly in \LaTeX\ is to use the
control sequence \verb/\,/ between the quotation marks,
so as to obtain the necessary amount of separation.  The above
example is thus produced with the input
\begin{quote}
\begin{verbatim}
``I regard computer typesetting as being reasonably
`straightforward'\,'' he said.
\end{verbatim}
\end{quote}

\subsection{Section Headings in \LaTeX}

Section headings of various sizes are produced (in the
\textbf{article} document style) using the commands
\verb/\section/,\verb/\subsection/ and
\verb/\subsubsection/
commands. \LaTeX\ will number the sections and subsections
automatically.  The title of the section should be surrounded
by curly brackets and placed immediately after the relevant
command.  Thus if we type
\begin{quote}
\begin{verbatim}
\section{Section Headings}

We explain in this section how to obtain headings
for the various sections and subsections of our
document.

\subsection{Headings in the `article' Document Style}

In the `article' style, the document may be divided up
into sections, subsections and subsubsections, and each
can be given a title, printed in a boldface font,
simply by issuing the appropriate command.
\end{verbatim}
\end{quote}
then the title of the section and that of the subsection
will be printed in a large boldface font, and will be
numbered accordingly.

Other document styles (such as the \textbf{book} and
\textbf{letter} styles) have other `sectioning'
commands available (for example, the \textbf{book}
style has a \verb/\chapter/ command for beginning
a new chapter).

Sometimes one wishes to suppress the automatic numbering
provided by \LaTeX.  This can be done by placing an
asterisk before the title of the section or subsection.
Thus, for example, the section numbers in the above example
could be suppressed by typing
\begin{quote}
\begin{verbatim}
\section*{Section Headings}

We explain in this section how to obtain headings
for the various sections and subsections of our
document.

\subsection*{Headings in the `article' Document Style}

In the `article' style, the document may be divided up
into sections, subsections and subsubsections, and each
can be given a title, printed in a boldface font,
simply by issuing the appropriate command.
\end{verbatim}
\end{quote}

\subsection{Changing Fonts in Text Mode}

\LaTeX\ has numerous commands for changing the typestyle.
The most useful of these is \verb/\emph{/\emph{text}\verb/}/
which \emph{emphasizes} some piece of text, setting it
usually in an \textit{italic font} (unless the surrounding
text is already italicized).  Thus for example, the text
\begin{quotation}
The basic results and techniques of \emph{Calculus}
were discovered and developed by \emph{Newton}
and \emph{Leibniz}, though many of the basic ideas
can be traced to earlier work of \emph{Cavalieri},
\emph{Fermat}, \emph{Barrow} and others.
\end{quotation}
is obtained by typing
\begin{quote}
\begin{verbatim}
The basic results and techniques of \emph{Calculus}
were discovered and developed by \emph{Newton}
and \emph{Leibniz}, though many of the basic ideas
can be traced to earlier work of \emph{Cavalieri},
\emph{Fermat}, \emph{Barrow} and others.
\end{verbatim}
\end{quote}

Another useful font-changing command is \verb/\textbf{/\emph{text}\verb/}/,
which typesets the specified portion of text in \textbf{boldface}.

A \emph{font family} or \emph{typeface} in \LaTeX\ consists of a
collection of related fonts characterized by \emph{size}, \emph{shape}
and \emph{series}.  The font families available in \LaTeX\ include
\textrm{roman}, \textsf{sans serif} and \texttt{typewriter}:
\begin{itemize}
\item
\textrm{Roman is normally the default family and includes
\textup{upright}, \textit{italic}, \textsl{slanted}, \textsc{small caps}
and \textbf{boldface} fonts of various sizes.}
\item
\textsf{There is a sans serif family with
\textup{upright}, \textsl{slanted}
and \textbf{boldface} fonts of various sizes.}
\item
\texttt{There is a typewriter family with
\textup{upright}, \textit{italic}, \textsl{slanted}\newline
and \textsc{small caps} fonts of various sizes.}
\end{itemize}

The sizes of fonts used in \LaTeX\ are can be determined
and changed by means of the control sequences
\verb/\tiny/,
\verb/\scriptsize/,
\verb/\footnotesize/,
\verb/\small/,
\verb/\normalsize/,
\verb/\large/,
\verb/\Large/,
\verb/\LARGE/,
\verb/\huge/ and
\verb/\HUGE/:
\begin{quote}
{\tiny This text is \texttt{tiny}}.

{\scriptsize This text is \texttt{scriptsize}}.

{\footnotesize This text is \texttt{footnotesize}}.

{\small This text is \texttt{small}}.

{\normalsize This text is \texttt{normalsize}}.

{\large This text is \texttt{large}}.

{\Large This text is \texttt{Large}}.

{\LARGE This text is \texttt{LARGE}}.

{\huge This text is \texttt{huge}}.

{\Huge This text is \texttt{Huge}}.
\end{quote}
\vspace{6pt}

The \emph{shape} of a font can be \textup{upright},
\textit{italic}, \textsl{slanted} or \textsc{small caps}:
\begin{itemize}
\item
\textup{The LaTeX command}
   \verb/\textup{/\emph{text}\verb/}/
   \textup{typesets the specified text with an upright shape:
   this is normally the default shape.}
\item
\textit{The LaTeX command}
   \verb/\textit{/\emph{text}\verb/}/
   \textit{typesets the specified text with an italic shape.}
\item
\textsl{The LaTeX command}
   \verb/\textsl{/\emph{text}\verb/}/
   \textsl{typesets the specified text with a slanted shape:
   slanted text is similar to italic.}
\item
\textsc{The LaTeX command}
   \verb/\textsc{/\emph{text}\verb/}/
   \textsc{typesets the specified text with a small caps shape
   in which all letters are capitals (with uppercase letters taller than
   lowercase letters).}
\end{itemize}

The \emph{series} of a font can be \textmd{medium} (the default) or
\textbf{boldface}:
\begin{itemize}
\item
\textup{The LaTeX command}
   \verb/\textmd{/\emph{text}\verb/}/
   \textmd{typesets the specified text with a medium series font.}
\item
\textbf{The LaTeX command}
   \verb/\textbf{/\emph{text}\verb/}/
   \textbf{typesets the specified text with a boldface series font.}
\end{itemize}
If the necessary fonts are available, one can combine changes to
the size, shape and series of a font, for example producing
\textbf{\textsl{boldface slanted text}} by typing
\begin{quote}
\begin{verbatim}
\textbf{\textsl{boldface slanted text}}.
\end{verbatim}
\end{quote}

There are in \LaTeX\ font declarations corresponding to the
the font-changing commands described above.  When included in the
\LaTeX\ input such declarations determine the type-style of the
subsequent text (till the next font declaration or the end
of the current `group' delimited by curly brackets
or by appropriate \verb/\begin/ and \verb/\end/
commands).  Here is a list of font-changing commands and
declarations in text mode:
\begin{quote}
\begin{tabular}{lll}
\emph{Command}&\emph{Declaration}&\\
\verb/\textrm/&\verb/\rmfamily/&{\rmfamily Roman family}\\
\verb/\textsf/&\verb/\sffamily/&{\sffamily Sans serif family}\\
\verb/\texttt/&\verb/\ttfamily/&{\ttfamily Typewriter family}\\[6pt]
\verb/\textup/&\verb/\upshape/&{\upshape Upright shape}\\
\verb/\textit/&\verb/\itshape/&{\itshape Italic shape}\\
\verb/\textsl/&\verb/\slshape/&{\slshape Slanted shape}\\
\verb/\textsc/&\verb/\scshape/&{\scshape Small caps shape}\\[6pt]
\verb/\textmd/&\verb/\mdseries/&{\mdseries Medium series}\\
\verb/\textbf/&\verb/\bfseries/&{\bfseries Boldface series}\\
\end{tabular}
\end{quote}

\subsection{Accents used in Text}

There are a variety of control sequences for producing accents.
For example, the control sequence \verb/\'{o}/ produces an
acute accent on the letter \verb/o/. Thus typing
\begin{quote}
\begin{verbatim}
Se\'{a}n \'{O} Cinn\'{e}ide.
\end{verbatim}
\end{quote}
produces
\begin{quotation}
Se\'{a}n \'{O} Cinn\'{e}ide.
\end{quotation}
Similarly we use the control sequence \verb/\`/ to
produce the grave accent in `alg\`{e}bre' and we use
\verb/\"/ to produce the umlaut in `Universit\"{a}t'.
The accents provided by \LaTeX\ include the following:
\begin{quote}
\begin{tabular}{lll}
\verb/\'{e}/  & \'{e}
           & e.g., \verb/math\'{e}matique/ yields `math\'{e}matique' \\
\verb/\`{e}/  & \`{e}
           & e.g., \verb/alg\`{e}bre/ yields `alg\`{e}bre' \\
\verb/\^{e}/  & \^{e}
           & e.g., \verb/h\^{o}te/ yields `h\^{o}te' \\
\verb/\"{o}/  & \"{o}
           & e.g., \verb/H\"{o}lder/ yields `H\"{o}lder' \\
\verb/\~{n}/  & \~{n}
           & e.g., \verb/ma\~{n}ana/ yields `ma\~{n}ana' \\
\verb/\={o}/  & \={o}  & \\
\verb/\.{o}/  & \.{o}  & \\
\verb/\u{o}/  & \u{o}  & \\
\verb/\v{c}/  & \v{c}
           & e.g., \verb/\v{C}ech/ yields `\v{C}ech' \\
\verb/\H{o}/  & \H{o}  & \\
\verb/\t{oo}/ & \t{oo} & \\
\verb/\c{c}/  & \c{c}
           & e.g., \verb/gar\c{c}on/ yields `gar\c{c}on' \\
\verb/\d{o}/  & \d{o}  & \\
\verb/\b{o}/  & \b{o}  &
\end{tabular} 
\end{quote}
These accents are for use in ordinary text. They cannot be
used within mathematical formulae, since different control 
sequences are used to produce accents within mathematics.

The control sequences \verb/\i/ and \verb/\j/ produce
dotless `i' and `j'.  These are required when placing an
accent on the letter. Thus \'{i} is produced by typing
\verb/\'{\i}/.

\subsection{Active Characters and Special Symbols in Text}

The `active characters'
\begin{quote}
\begin{verbatim}
# $ % & \ ^ _ { } ~
\end{verbatim}
\end{quote}
have special purposes within \LaTeX.  Thus they cannot be produced
in the  final document simply by typing them directly.  On the
rare occasions when one needs to use the special characters
\begin{quote}
\#\ \$\ \%\ \&\ \_\ \{\ \}
\end{quote}
in the final document, they can be produced by typing the control
sequences
\begin{quote}
\begin{verbatim}
\# \$ \% \& \_ \{ \}
\end{verbatim}
\end{quote}
respectively.  However the characters
\verb/\/, \verb/^/ and \verb/~/ cannot be
produced simply by preceding them with a backslash.  They can
however be produced using \verb/\char92/ (in the \verb/\texttt/ font
only), \verb/\char94/ and \verb/\char126/ respectively.
(The decimal numbers 92, 94 and 126 are the ASCII codes of these
characters.)

Other special symbols can be introduced into
text using the appropriate control sequences:
\begin{quote}
\begin{tabular}{l|l}
\emph{Symbol} & \emph{Control Sequence}\\
\hline
\oe, \OE   & \verb/\oe, \OE/   \\
\ae, \AE   & \verb/\ae, \AE/   \\
\aa, \AA   & \verb/\aa, \AA/   \\
\o, \O     & \verb/\o, \O/     \\
\l, \L     & \verb/\l, \L/     \\
\ss        & \verb/\ss/        \\
?`         & \verb/?`/         \\
!`         & \verb/!`/         \\
\dag       & \verb/\dag/       \\
\ddag      & \verb/\ddag/      \\
\S         & \verb/\S/         \\
\P         & \verb/\P/         \\
\copyright & \verb/\copyright/ \\
\pounds    & \verb/\pounds/    \\
\i         & \verb/\i/         \\
\j         & \verb/\j/         
\end{tabular}
\end{quote}

\section{Producing Mathematical Formulae using \LaTeX}

\subsection{Mathematics Mode}

In order to obtain a mathematical formula using \LaTeX, one must
enter \emph{mathematics mode} before the formula and leave it
afterwards. Mathematical formulae can occur either embedded in text
or else displayed between lines of text. When a formula occurs within
the text of a paragraph one should place a \verb/$/ sign before and
after the formula, in order to enter and leave mathematics mode.
Thus to obtain a sentence like
\begin{quotation}
Let $f$ be the function defined by $f(x) = 3x + 7$, and
let $a$ be a positive real number.
\end{quotation}
one should type
\begin{quote}
\begin{verbatim}
Let $f$ be the function defined by $f(x) = 3x + 7$, and
let $a$ be a positive real number.
\end{verbatim}
\end{quote}
In particular, note that even mathematical expressions consisting
of a single character, like $f$ and $a$ in the example above,
are placed within \verb/$/ signs. This is to ensure that they are set
in italic type, as is customary in mathematical typesetting.

\LaTeX\ also allows you to use \verb/\(/ and \verb/\)/ to mark
the beginning and the end respectively of a mathematical formula
embedded in text. Thus
\begin{quotation}
Let \( f \) be the function defined by \( f(x) = 3x + 7 \).
\end{quotation}
may be produced by typing
\begin{quote}
\begin{verbatim}
Let \( f \) be the function defined by \( f(x) = 3x + 7 \).
\end{verbatim}
\end{quote}
However this use of \verb/\(/ ... \verb/\)/ is only
permitted in \LaTeX: other dialects of TeX such as
Plain \TeX\ and AmSTeX use \verb/$/ ... \verb/$/.

In order to obtain a mathematical formula or equation which
is displayed on a line by itself, one places \verb/\[/
before and
\verb/\]/ after the formula. Thus to obtain
\begin{quotation}
If $f(x) = 3x + 7$ and $g(x) = x + 4$ then
\[ f(x) + g(x) = 4x + 11 \] 
and
\[ f(x)g(x) = 3x^2 + 19x +28. \]
\end{quotation}
one would type
\begin{quote}
\begin{verbatim}
If $f(x) = 3x + 7$ and $g(x) = x + 4$ then
\[ f(x) + g(x) = 4x + 11 \] 
and
\[ f(x)g(x) = 3x^2 + 19x +28. \]
\end{verbatim}
\end{quote}
(Here the character \verb/^/ is used to obtain a superscript.)

\LaTeX\ provides facilities for the automatic numbering of
displayed equations. If you want an numbered equation then you
use \verb/\begin{equation}/ and \verb/\end{equation}/ instead
of using \verb/\[/ and \verb/\]/ . Thus
\begin{quote}
\begin{verbatim}
If $f(x) = 3x + 7$ and $g(x) = x + 4$ then
\begin{equation}
f(x) + g(x) = 4x + 11
\end{equation}
and
\begin{equation}
f(x)g(x) = 3x^2 + 19x +28.
\end{equation}
\end{verbatim}
\end{quote}
produces
\begin{quotation}
If $f(x) = 3x + 7$ and $g(x) = x + 4$ then
\begin{equation}
f(x) + g(x) = 4x + 11
\end{equation}
and
\begin{equation}
f(x)g(x) = 3x^2 + 19x +28.
\end{equation}
\end{quotation}

\subsection{Characters in Mathematics Mode}

All the characters on the keyboard have their standard meaning
in mathematics mode, with the exception of the characters
\begin{quote}
\begin{verbatim}
      # $ % & ~ _ ^ \ { } '
\end{verbatim}
\end{quote}
Letters are set in italic type. In mathematics mode the character
\verb/'/ has a special meaning: typing \verb/$u' + v''$/
produces 
$u' + v''$
When in mathematics mode the spaces you type
between letters and other symbols do not affect the spacing of
the final result, since \LaTeX\ determines the spacing of characters
in formulae by its own internal rules. Thus \verb/$u v + w = x$/
and \verb/$uv+w=x$/ both produce 
$u v + w = x$
You can also type carriage returns where necessary in your input file
(e.g., if you are typing in a complicated formula with many
Greek characters and funny symbols) and this will have no effect on
the final result if you are in mathematics mode.

To obtain the characters
\begin{quote}
\begin{verbatim}
      #   $   %   &   _   {   }
\end{verbatim}
\end{quote}
in mathematics mode, one should type
\begin{quote}
\begin{verbatim}
      \#   \$   \%   \&   \_   \{   \} .
\end{verbatim}
\end{quote}
To obtain \verb/\/ in mathematics mode, one may type
\verb/\backslash/.

\subsection{Superscripts and Subscripts}

Subscripts and superscripts are obtained using the special
characters \verb/_/ and \verb/^/ respectively. Thus the
identity 
\[ ds^2 = dx_1^2 + dx_2^2 + dx_3^2 - c^2 dt^2 \]
is obtained by typing
\begin{quote}
\begin{verbatim}
\[ ds^2 = dx_1^2 + dx_2^2 + dx_3^2 - c^2 dt^2 \]
\end{verbatim}
\end{quote}
It can also be obtained by typing
\begin{quote}
\begin{verbatim}
\[ ds^2 = dx^2_1 + dx^2_2 + dx^2_3 - c^2 dt^2 \]
\end{verbatim}
\end{quote}
since, when a superscript is to appear above a subscript, it is
immaterial whether the superscript or subscript is the first to be
specified.

Where more than one character occurs in a superscript or subscript,
the characters involved should be enclosed in curly brackets.
For example, the polynomial
$x^{17} - 1$
is obtained by typing \verb/$x^{17} - 1$/.

One may not type expressions such as \verb/$s^n^j$/ since this is
ambiguous and could be interpreted either as
$s^{n j}$
or as 
$s^{n^j}$
The first of these alternatives is
obtained by typing \verb/$s^{n j}$/, the second by typing
\verb/$s^{n^j}$/.  A similar remark applies to subscripts.
Note that one can obtain in this way double superscripts (where a
superscript is placed on a superscript) and double subscripts.

It is sometimes necessary to obtain expressions in which the horizontal
ordering of the subscripts is significant.  One can use an `empty group'
\verb/{}/ to separate superscripts and subscripts that must follow
one another.  For example, the identity
\[ R_i{}^j{}_{kl} = g^{jm} R_{imkl}
   = - g^{jm} R_{mikl} = - R^j{}_{ikl} \]
can be obtained by typing
\begin{quote}
\begin{verbatim}
\[ R_i{}^j{}_{kl} = g^{jm} R_{imkl}
   = - g^{jm} R_{mikl} = - R^j{}_{ikl} \]
\end{verbatim}
\end{quote}

\subsection{Greek Letters}

Greek letters are produced in mathematics mode by preceding the
name of the letter by a backslash \verb/\/.  Thus to
obtain the formula
$A = \pi r^2$
one types \verb/A = \pi r^2/.

Here are the control sequences for the standard forms of the
lowercase Greek letters:-

\begin{quotation}
\mathlexicon{alpha&iota&rho\cr
beta&kappa&sigma\cr
gamma&lambda&tau\cr
delta&mu&upsilon\cr
epsilon&nu&phi\cr
zeta&xi&chi\cr
eta&\omit\qquad \rlap{$o$}\qquad \texttt{o}\hfil&psi\cr
theta&pi&omega\cr}
\end{quotation}

There is no special command for omicron: just use \verb/o/.

Some Greek letters occur in variant forms. The variant forms
are obtained by preceding the name of the Greek letter by `var'.
The following table lists the usual form of these letters and
the variant forms:-
\begin{quotation}
$$\vcenter{\halign{\displayandname{#}\hfil&&\qquad
                   \displayandname{#}\hfil\cr
epsilon&varepsilon\cr
theta&vartheta\cr
pi&varpi\cr
rho&varrho\cr
sigma&varsigma\cr
phi&varphi\cr}}$$
\end{quotation}

Upper case Greek letters are obtained by making the first character
of the name upper case.  Here are the control sequence for the
uppercase letters:---

\begin{quotation}
\mathlexicon{Gamma&Xi&Phi\cr
Delta&Pi&Psi\cr
Theta&Sigma&Omega\cr
Lambda&Upsilon&\omit\hfil\cr}
\end{quotation}

\subsection{Mathematical Symbols}

There are numerous mathematical symbols that can be used in
mathematics mode. These are obtained by typing an appropriate
control sequence.

Miscellaneous Symbols:

\begin{quotation}
\mathlexicon{aleph&prime&forall\cr
hbar&emptyset&exists\cr
imath&nabla&neg\cr
jmath&surd&flat\cr
ell&top&natural\cr
wp&bot&sharp\cr
Re&|&clubsuit\cr
Im&angle&diamondsuit\cr
partial&triangle&heartsuit\cr
infty&backslash&spadesuit\cr}
\end{quotation}

``Large'' Operators:

\begin{quotation}
\mathlexicon{sum&bigcap&bigodot\cr
prod&bigcup&bigotimes\cr
coprod&bigsqcup&bigoplus\cr
int&bigvee&biguplus\cr
oint&bigwedge&\omit\hfil\cr}
\end{quotation}

Binary Operations:

\begin{quotation}
\mathlexicon{pm&cap&vee\cr
mp&cup&wedge\cr
setminus&uplus&oplus\cr
cdot&sqcap&ominus\cr
times&sqcup&otimes\cr
ast&triangleleft&oslash\cr
star&triangleright&odot\cr
diamond&wr&dagger\cr
circ&bigcirc&ddagger\cr
bullet&bigtriangleup&amalg\cr
div&bigtriangledown&\omit\hfil\cr}
\end{quotation}

Relations:

\begin{quotation}
\mathlexicon{leq&geq&equiv\cr
prec&succ&sim\cr
preceq&succeq&simeq\cr
ll&gg&asymp\cr
subset&supset&approx\cr
subseteq&supseteq&cong\cr
sqsubseteq&sqsupseteq&bowtie\cr
in&ni&propto\cr
vdash&dashv&models\cr
smile&mid&doteq\cr
frown&parallel&perp\cr}
\end{quotation}

Negated Relations:

\begin{quotation}
\def\negdisplayandname#1{\rlap{$\displaystyle\not\csname #1\endcsname$}%
\qquad \texttt{\char92 not\char92 #1}}
$$\vcenter{\halign{\negdisplayandname{#}\hfil&&\qquad
        \negdisplayandname{#}\hfil\cr
\omit\rlap{$\not<$}\qquad\texttt{\char92 not<}\hfil&\omit
\qquad\rlap{$\not>$}\qquad\texttt{\char92 not>}\hfil&\omit
\qquad\rlap{$\not=$}\qquad\texttt{\char92 not=}\hfil\cr
leq&geq&equiv\cr
prec&succ&sim\cr
preceq&succeq&simeq\cr
subset&supset&approx\cr
subseteq&supseteq&cong\cr
sqsubseteq&sqsupseteq&asymp\cr}}$$
\end{quotation}

Arrows:

\begin{quotation}
\mathlexicon{%
leftarrow&rightarrow\cr
longleftarrow&longrightarrow\cr
Leftarrow&Rightarrow\cr
Longleftarrow&Longrightarrow\cr
leftrightarrow&Leftrightarrow\cr
longleftrightarrow&Longleftrightarrow\cr
hookleftarrow&hookrightarrow\cr
leftharpoonup&rightharpoonup\cr
leftharpoondown&rightharpoondown\cr
uparrow&downarrow\cr
Uparrow&Downarrow\cr
updownarrow&Updownarrow\cr
nearrow&nwarrow\cr
searrow&swarrow\cr
mapsto&longmapsto\cr
rightleftharpoons&\omit\hfil\cr}
\end{quotation}

Openings:

\begin{quotation}
\mathlexicon{lbrack&lfloor&lceil\cr
lbrace&langle&\omit\hfil\cr}
\end{quotation}

Closings:

\begin{quotation}
\mathlexicon{rbrack&rfloor&rceil\cr
rbrace&rangle&\omit\hfil\cr}
\end{quotation}

Alternative Names:

\begin{quotation}
\def\widedisplayandname#1{\rlap{$\displaystyle\csname #1\endcsname$}%
                      \qquad\qquad \texttt{\char92 #1}}
$$\vcenter{\halign{\widedisplayandname{#}\hfil&\qquad
(same as \texttt{\char92 #})\hfil\cr
\omit\rlap{$\not=$}\qquad\qquad
      \texttt{\char92 ne} or \texttt{\char92 neq}\hfil&not=\cr
le&leq\cr
ge&geq\cr
\omit\rlap{$\{$}\qquad\qquad\texttt{\char92 \char123}\hfil&lbrace\cr
\omit\rlap{$\}$}\qquad\qquad\texttt{\char92 \char125}\hfil&lbrace\cr
to&rightarrow\cr
gets&leftarrow\cr
owns&ni\cr
land&wedge\cr
lor&vee\cr
lnot&neg\cr
vert&\omit\qquad (same as \texttt{|})\hfil\cr
Vert&\omit\qquad (same as \texttt{\char92 |})\hfil\cr
iff&\omit\qquad (same as \texttt{\char92 Longleftrightarrow}, but with\hfil\cr
\omit\hfil&\omit\qquad\ extra space at each end)\hfil\cr
colon&\omit\qquad (same as \texttt{:}, but with less space around it and\hfil\cr
\omit\hfil&\omit
\qquad\ less likelihood of a line break after it)\hfil\cr}}$$
\end{quotation}

\subsection{Changing Fonts in Mathematics Mode}

\emph{(The following applies to \LaTeX2$\epsilon$, a
recent version of \LaTeX.  It does not apply to older
versions of \LaTeX.)}

The `math italic' font is automatically
used in mathematics mode unless you explicitly change the font.
The rules for changing the font in mathematics mode are rather different
to those applying when typesetting ordinary text. 
In mathematics mode any change only applies to the single character
or symbol that follows (or to any text enclosed within curly brackets
immediately following the control sequence).  Also, to change
a character to the roman or boldface font, the control sequences
\verb/\mathrm/ and \verb/\mathbf/ must be used
(rather than \verb/\textrm/ and \verb/\textbf/).

The following example illustrates the use of boldface in mathematical
formulae.  To obtain
\begin{quotation}
Let $\mathbf{u}$,$\mathbf{v}$ and $\mathbf{w}$ be three
vectors in ${\mathbf R}^3$. The volume~$V$ of the
parallelepiped with corners at the points
$\mathbf{0}$, $\mathbf{u}$, $\mathbf{v}$,
$\mathbf{w}$, $\mathbf{u}+\mathbf{v}$,
$\mathbf{u}+\mathbf{w}$, $\mathbf{v}+\mathbf{w}$
and $\mathbf{u}+\mathbf{v}+\mathbf{w}$
is given by the formula
\[ V = (\mathbf{u} \times \mathbf{v}) \cdot \mathbf{w}.\] 
\end{quotation}
one could type
\begin{quote}
\begin{verbatim}
Let $\mathbf{u}$,$\mathbf{v}$ and $\mathbf{w}$ be three
vectors in ${\mathbf R}^3$. The volume~$V$ of the
parallelepiped with corners at the points
$\mathbf{0}$, $\mathbf{u}$, $\mathbf{v}$,
$\mathbf{w}$, $\mathbf{u}+\mathbf{v}$,
$\mathbf{u}+\mathbf{w}$, $\mathbf{v}+\mathbf{w}$
and $\mathbf{u}+\mathbf{v}+\mathbf{w}$
is given by the formula
\[ V = (\mathbf{u} \times \mathbf{v}) \cdot \mathbf{w}.\] 
\end{verbatim}
\end{quote}

There is also a `calligraphic' font available in mathematics mode.
This is obtained using the control sequence \verb/\cal/. 
\emph{This font can only be used for uppercase letters.} 
These calligraphic letters have the form
\[ \cal{A}\cal{B}\cal{C}\cal{D}\cal{E}\cal{F}\cal{G}\cal{H}\cal{I}
\cal{J}\cal{K}\cal{L}\cal{M}\cal{N}\cal{O}\cal{P}\cal{Q}\cal{R}
\cal{S}\cal{T}\cal{U}\cal{V}\cal{W}\cal{X}\cal{Y}\cal{Z}. \]

\subsection{Standard Functions (sin, cos etc.)}

The names of certain standard functions and abbreviations are
obtained by typing a backlash \verb/\/ before the name.
For example, one obtains
\[ \cos(\theta + \phi) = \cos \theta \cos \phi
      - \sin \theta \sin \phi \]
by typing
\begin{quote}
\begin{verbatim}
\[ \cos(\theta + \phi) = \cos \theta \cos \phi
      - \sin \theta \sin \phi \]
\end{verbatim}
\end{quote}

The following standard functions are represented by control sequences
defined in \LaTeX:
\[ 
\vcenter{\halign{$\backslash$\texttt{#}&&\quad 
$\backslash$\texttt{#}\cr
arccos&cos&csc&exp&ker&limsup&min&sinh\cr
arcsin&cosh&deg&gcd&lg&ln&Pr&sup\cr
arctan&cot&det&hom&lim&log&sec&tan\cr
arg&coth&dim&inf&liminf&max&sin&tanh\cr}} \]

Names of functions and other abbreviations not in this list can be
obtained by converting to the roman font. Thus one obtains
$\mathrm{cosec} A$
by typing \verb/$\mathrm{cosec} A$/.
Note that if one were to type simply \verb/$cosec A$/ one
would obtain
$cosec A$,
because \LaTeX\ has treated
\verb/cosec A/ as the product of six quantities
$c$, $o$, $s$, $e$, $c$ and $A$
and typeset the formula accordingly.

\subsection{Text Embedded in Displayed Equations}

Text can be embedded in displayed equations (in \LaTeX) by using
\verb/\mbox{/\emph{embedded text}\verb/}/.  For example,
one obtains
\[ M^\bot = \{ f \in V' : f(m) = 0 \mbox{ for all } m \in M \}. \]
by typing
\begin{quote}
\begin{verbatim}
\[ M^\bot = \{ f \in V' : f(m) = 0 \mbox{ for all } m \in M \}.\] 
\end{verbatim}
\end{quote}
Note the blank spaces before and after the words `for all' in the above
example. Had we typed
\begin{quote}
\begin{verbatim}
\[ M^\bot = \{ f \in V' : f(m) = 0 \mbox{for all} m \in M \}.\] 
\end{verbatim}
\end{quote}
we would have obtained
\[ M^\bot = \{ f \in V' : f(m) = 0 \mbox{for all} m \in M \}. \]

(In Plain \TeX\ one should use \verb/\hbox/ in place of
\verb/\mbox/.)

\subsection{Fractions and Roots}

Fractions of the form
\[ \frac{\mbox{\textit{numerator}}}{\mbox{\textit{denominator}}} \]
are obtained in \LaTeX\ using the construction
\begin{quote}
\verb/\frac{/\emph{numerator}\verb/}{/\emph{denominator}\verb/}/.
\end{quote}
For example, to obtain
\begin{quotation}
The function $f$ is given by
\[ f(x) = 2x + \frac{x - 7}{x^2 + 4}\] 
for all real numbers $x$.
\end{quotation}
one would type
\begin{quote}
\begin{verbatim}
The function $f$ is given by
\[ f(x) = 2x + \frac{x - 7}{x^2 + 4}\] 
for all real numbers $x$.
\end{verbatim}
\end{quote}

To obtain square roots one uses the control sequence
\begin{quote}
\verb/\sqrt{/\emph{expression}\verb/}/.
\end{quote}
For example, to obtain
\begin{quotation}
The roots of a quadratic polynomial $a x^2 + bx + c$ with
$a \neq 0$ are given by the formula
\[ \frac{-b \pm \sqrt{b^2 - 4ac}}{2a} \]
\end{quotation}
one would type
\begin{quote}
\begin{verbatim}
The roots of a quadratic polynomial $a x^2 + bx + c$ with
$a \neq 0$ are given by the formula
\[ \frac{-b \pm \sqrt{b^2 - 4ac}}{2a} \]
\end{verbatim}
\end{quote}

In \LaTeX, an 
$n$th root is produced using
\begin{quote}
\verb/\sqrt[n]{/\emph{expression}\verb/}/.
\end{quote}
For example, to obtain
\begin{quotation}
The roots of a cubic polynomial of the form $x^3 - 3px - 2q$
are given by the formula
\[ \sqrt[3]{q + \sqrt{ q^2 - p^3 }}
  + \sqrt[3]{q - \sqrt{ q^2 - p^3 }} \]
where the values of the two cube roots must are chosen
so as to ensure that their product is equal to $p$.
\end{quotation}
in \LaTeX, one would type
\begin{quote}
\begin{verbatim}
The roots of a cubic polynomial of the form $x^3 - 3px - 2q$
are given by the formula
\[ \sqrt[3]{q + \sqrt{ q^2 - p^3 }}
  + \sqrt[3]{q - \sqrt{ q^2 - p^3 }} \]
where the values of the two cube roots must are chosen
so as to ensure that their product is equal to $p$.
\end{verbatim}
\end{quote}

\subsection{Ellipsis (i.e., `three dots')}

Ellipsis (three dots) is produced in mathematics mode using
the control sequences \verb/\ldots/ (for dots aligned
with tbe baseline of text), and \verb/\cdots/ (for dots
aligned with the centreline of mathematical formulae).  Thus
the formula
\[ f(x_1, x_2,\ldots, x_n) = x_1^2 + x_2^2 + \cdots + x_n^2 \]
is obtained by typing
\begin{quote}
\begin{verbatim}
\[ f(x_1, x_2,\ldots, x_n) = x_1^2 + x_2^2 + \cdots + x_n^2 \]
\end{verbatim}
\end{quote}
Similarly the formula
\[ \frac{1 - x^{n+1}}{1 - x} = 1 + x + x^2 + \cdots + x^n \]
is produced using \verb/\cdots/, by typing
\begin{quote}
\begin{verbatim}
\[ \frac{1 - x^{n+1}}{1 - x} = 1 + x + x^2 + \cdots + x^n \]
\end{verbatim}
\end{quote}

\subsection{Accents in Mathematics Mode}

There are various control sequences for producing underlining,
overlining and various accents in mathematics mode.  The
following table lists these control sequences, applying them
to the letter~$a$:
\begin{quotation}
\def\exhibitaccent#1#2{\rlap{$\csname #1\endcsname{#2}$}%
                      \qquad \texttt{\char92 #1\char123 #2\char125}}
$$\vcenter{\halign{#\hfil\cr
\exhibitaccent{underline}{a}\cr
\exhibitaccent{overline}{a}\cr
\exhibitaccent{hat}{a}\cr
\exhibitaccent{check}{a}\cr
\exhibitaccent{tilde}{a}\cr
\exhibitaccent{acute}{a}\cr
\exhibitaccent{grave}{a}\cr
\exhibitaccent{dot}{a}\cr
\exhibitaccent{ddot}{a}\cr
\exhibitaccent{breve}{a}\cr
\exhibitaccent{bar}{a}\cr
\exhibitaccent{vec}{a}\cr
}}$$
\end{quotation}
It should be borne in mind that when a character is underlined in
a mathematical manuscript then it is normally typeset in
bold face without any underlining. Underlining is used very
rarely in print.

The control sequences such as \verb/\'/ and \verb/\"/, used
to produce accents in ordinary text, may not be used in
mathematics mode.

\subsection{Brackets and Norms}

The frequently used left delimiters include (, [ and \{,
which are obtained by typing \verb/(/, \verb/[/ and
\verb/\{/
respectively. The corresponding right delimiters are of course 
obtained by typing \verb/)/,
\verb/]/ and \verb/\}/.  In addition 
$|$
and 
$\|$
are used as both left and right delimiters, and are obtained by typing
\verb/|/ and \verb/\|/ respectively. For example, we obtain
\begin{quotation}
Let $X$ be a Banach space and let $f \colon B \to \textbf{R}$
be a bounded linear functional on $X$. The \textit{norm} of
$f$, denoted by $\|f\|$, is defined by
\[ \|f\| = \inf \{ K \in [0,+\infty) :
          |f(x)| \leq K \|x\| \mbox{ for all } x \in X \}.\] 
\end{quotation}
by typing
\begin{quote}
\begin{verbatim}
Let $X$ be a Banach space and let $f \colon B \to \textbf{R}$
be a bounded linear functional on $X$. The \textit{norm} of
$f$, denoted by $\|f\|$, is defined by
\[ \|f\| = \inf \{ K \in [0,+\infty) :
          |f(x)| \leq K \|x\| \mbox{ for all } x \in X \}.\] 
\end{verbatim}
\end{quote}

Larger delimiters are sometimes required which have the
appropriate height to match the size of the subformula which
they enclose. Consider, for instance, the problem of typesetting
the following formula:
\[ f(x,y,z) = 3y^2 z \left( 3 + \frac{7x+5}{1 + y^2} \right). \]
The way to type the large parentheses is to type \verb/\left(/
for the left parenthesis and \verb/\right)/ for the right
parenthesis, and let \LaTeX\ do the rest of the work for you.
Thus the above formula was obtained by typing
\begin{quote}
\begin{verbatim}
\[ f(x,y,z) = 3y^2 z \left( 3 + \frac{7x+5}{1 + y^2} \right).\] 
\end{verbatim}
\end{quote}
If you type a delimiter which is preceded by \verb/\left/ then
\LaTeX\ will search for a corresponding delimiter preceded by
\verb/\right/ and calculate the size of the delimiters required
to enclose the intervening subformula. One is allowed to balance
a \verb/\left(/ with a \verb/\right]/ (say) if one desires: there
is no reason why the enclosing delimiters have to have the same
shape. One may also nest pairs of delimiters within one another:
by typing
\begin{quote}
\begin{verbatim}
\[ \left| 4 x^3 + \left( x + \frac{42}{1+x^4} \right) \right|.\] 
\end{verbatim}
\end{quote}
we obtain
\[ \left| 4 x^3 + \left( x + \frac{42}{1+x^4} \right) \right|. \]

By typing \verb/\left./ and \verb/\right./ one obtains
\emph{null delimiters} which are completely invisible. Consider,
for example, the problem of typesetting
\[ \left. \frac{du}{dx} \right|_{x=0}.\] 
We wish to make the vertical bar big enough to match the
derivative preceding it. To do this, we suppose that the
derivative is enclosed by delimiters, where the left delimiter
is invisible and the right delimiter is the vertical line.
The invisible delimiter is produced using \verb/\left./ and thus
the whole formula is produced by typing
\[ \left. \frac{du}{dx} \right|_{x=0}.\] 

\subsection{Multiline Formulae in \LaTeX}

Consider the problem of typesetting the formula
\begin{quotation}
\begin{eqnarray*}
\cos 2\theta & = & \cos^2 \theta - \sin^2 \theta \\
             & = & 2 \cos^2 \theta - 1.
\end{eqnarray*}
\end{quotation}
It is necessary to ensure that the = signs are aligned with one
another. In \LaTeX, such a formula is typeset using the
\verb/eqnarray*/ environment. The above example was obtained by
typing the lines
\begin{quote}
\begin{verbatim}
\begin{eqnarray*}
\cos 2\theta & = & \cos^2 \theta - \sin^2 \theta \\
             & = & 2 \cos^2 \theta - 1.
\end{eqnarray*}
\end{verbatim}
\end{quote}
Note the use of the special character \verb/&/ as an
\emph{alignment tab}. When the formula is typeset, the part of
the second line of the formula beginning with an occurrence of
\verb/&/ will be placed immediately beneath that part of
the first line of the formula which begins with the corresponding
occurrence of \verb/&/.  Also \verb/\\/ is used to
separate the lines of the formula.

Although we have placed corresponding occurrences of \verb/&/
beneath one another in the above example, it is not necessary to
do this in the input file. It was done in the above example merely
to improve the appearance (and readability) of the input file.

The more complicated example
\begin{quotation}
If $h \leq \frac{1}{2} |\zeta - z|$ then
\[ |\zeta - z - h| \geq \frac{1}{2} |\zeta - z|\] 
and hence
\begin{eqnarray*}
\left| \frac{1}{\zeta - z - h} - \frac{1}{\zeta - z} \right|
& = & \left|
\frac{(\zeta - z) - (\zeta - z - h)}{(\zeta - z - h)(\zeta - z)}
\right| \\  & = &
\left| \frac{h}{(\zeta - z - h)(\zeta - z)} \right| \\
  & \leq & \frac{2 |h|}{|\zeta - z|^2}.
\end{eqnarray*}
\end{quotation}
was obtained by typing
\begin{quote}
\begin{verbatim}
If $h \leq \frac{1}{2} |\zeta - z|$ then
\[ |\zeta - z - h| \geq \frac{1}{2} |\zeta - z|\] 
and hence
\begin{eqnarray*}
\left| \frac{1}{\zeta - z - h} - \frac{1}{\zeta - z} \right|
& = & \left|
\frac{(\zeta - z) - (\zeta - z - h)}{(\zeta - z - h)(\zeta - z)}
\right| \\  & = &
\left| \frac{h}{(\zeta - z - h)(\zeta - z)} \right| \\
  & \leq & \frac{2 |h|}{|\zeta - z|^2}.
\end{eqnarray*}
\end{verbatim}
\end{quote}

The asterisk in \verb/eqnarray*/ is put there to suppress the
automatic equation numbering produced by \LaTeX. If you wish for
an automatically numbered multiline formula, you should use
\verb/\begin{eqnarray}/ and \verb/\end{eqnarray}/.

\subsection{Matrices and other arrays in \LaTeX}

Matrices and other arrays are produced in \LaTeX\ using the
\textbf{array} environment. For example, suppose that we wish to
typeset the following passage:
\begin{quotation}
The \emph{characteristic polynomial} $\chi(\lambda)$ of the
$3 \times 3$~matrix
\[ \left( \begin{array}{ccc}
a & b & c \\
d & e & f \\
g & h & i \end{array} \right)\] 
is given by the formula
\[ \chi(\lambda) = \left| \begin{array}{ccc}
\lambda - a & -b & -c \\
-d & \lambda - e & -f \\
-g & -h & \lambda - i \end{array} \right|.\] 
\end{quotation}
This passage is produced by the following input:
\begin{quote}
\begin{verbatim}
The \emph{characteristic polynomial} $\chi(\lambda)$ of the
$3 \times 3$~matrix
\[ \left( \begin{array}{ccc}
a & b & c \\
d & e & f \\
g & h & i \end{array} \right)\] 
is given by the formula
\[ \chi(\lambda) = \left| \begin{array}{ccc}
\lambda - a & -b & -c \\
-d & \lambda - e & -f \\
-g & -h & \lambda - i \end{array} \right|.\] 
\end{verbatim}
\end{quote}
First of all, note the use of \verb/\left/ and \verb/\right/
to produce the large delimiters around the arrays. As we have
already seen, if we use
\begin{quote}
\verb/\left)/ ... \verb/\right)/
\end{quote}
then the size of the parentheses is chosen to match the subformula
that they enclose. Next note the use of the alignment tab
character \verb/&/ to separate the entries of the matrix and
the use of \verb/\\/ to separate the rows of the matrix, exactly
as in the construction of multiline formulae described above.
We begin the array with \verb/\begin{array}/ and end it with
\verb/\end{array}/. The only thing left to explain, therefore,
is the mysterious \verb/{ccc}/ which occurs immediately after
\verb/\begin{array}/.
Now each of the \verb/c/'s in \verb/{ccc}/ represents a
column of the matrix and indicates that the entries of the
column should be \emph{centred}. If the \verb/c/ were replaced by
\verb/l/ then the corresponding column would be typeset with
all the entries flush \emph{left}, and \verb/r/ would produce a
column with all entries flush \emph{right}. Thus
\begin{quote}
\begin{verbatim}
\[ \begin{array}{lcr}
\mbox{First number} & x & 8 \\
\mbox{Second number} & y & 15 \\
\mbox{Sum} & x + y & 23 \\
\mbox{Difference} & x - y & -7 \\
\mbox{Product} & xy & 120 \end{array}\] 
\end{verbatim}
\end{quote}
produces
\[ \begin{array}{lcr}
\mbox{First number} & x & 8 \\
\mbox{Second number} & y & 15 \\
\mbox{Sum} & x + y & 23 \\
\mbox{Difference} & x - y & -7 \\
\mbox{Product} & xy & 120 \end{array}\] 

We can use the array environment to produce formulae such as
\[ |x| = \left\{ \begin{array}{ll}
         x & \mbox{if $x \geq 0$};\\
        -x & \mbox{if $x < 0$}.\end{array} \right.  \]
Note that both columns of this array are set flush left. Thus we
use \verb/{ll}/ immediately after \verb/\begin{array}/. The large
curly bracket is produced using \verb/\left\{/. However this
requires a corresponding \verb/\right/ delimiter to match it.
We therefore use the \emph{null delimiter} \verb/\right./
discussed earlier. This delimiter is invisible. We can
therefore obtain the above formula by typing
\begin{quote}
\begin{verbatim}
\[ |x| = \left\{ \begin{array}{ll}
         x & \mbox{if $x \geq 0$};\\
        -x & \mbox{if $x < 0$}.\end{array} \right. \] 
\end{verbatim}
\end{quote}

\subsection{Derivatives, Limits, Sums and Integrals}

The expressions
\[ \frac{du}{dt} \mbox{ and } \frac{d^2 u}{dx^2} \]
are obtained in \LaTeX\ by typing \verb/\frac{du}{dt}/
and \verb/\frac{d^2 u}{dx^2}/ respectively. The mathematical
symbol 
$\partial$
is produced using \verb/\partial/. Thus the Heat Equation
\[\frac{\partial u}{\partial t}
   = \frac{\partial^2 u}{\partial x^2}
      + \frac{\partial^2 u}{\partial y^2}
      + \frac{\partial^2 u}{\partial z^2} \]
is obtained in \LaTeX\ by typing
\begin{quote}
\begin{verbatim}
\[\frac{\partial u}{\partial t}
   = \frac{\partial^2 u}{\partial x^2}
      + \frac{\partial^2 u}{\partial y^2}
      + \frac{\partial^2 u}{\partial z^2} \]
\end{verbatim}
\end{quote}

To obtain mathematical expressions such as
\[ \lim_{x \to +\infty} \mbox{, } \inf_{x > s} \mbox{ and } \sup_K \]
in displayed equations we type \verb/\lim_{x \to +\infty}/,
\verb/\inf_{x > s}/ and \verb/\sup_K/ respectively. Thus to obtain
\[ \lim_{x \to 0} \frac{3x^2 +7}{x^2 +1} = 3. \]
(in \LaTeX) we type
\begin{quote}
\begin{verbatim}
\[ \lim_{x \to 0} \frac{3x^2 +7x^3}{x^2 +5x^4} = 3.\] 
\end{verbatim}
\end{quote}

To obtain a summation sign such as
\[ \sum_{i=1}^{2n} \]
we type \verb/sum_{i=1}^{2n}/. Thus
\[ \sum_{k=1}^n k^2 = \frac{1}{2} n (n+1). \]
is obtained by typing
\begin{quote}
\begin{verbatim}
\[ \sum_{k=1}^n k^2 = \frac{1}{2} n (n+1).\] 
\end{verbatim}
\end{quote}

We now discuss how to obtain \emph{integrals} in mathematical
documents. A typical integral is the following:
\[ \int_a^b f(x)\,dx. \]
This is typeset using
\begin{quote}
\begin{verbatim}
\[ \int_a^b f(x)\,dx.\] 
\end{verbatim}
\end{quote}
The integral sign 
$\int$
is typeset using the control sequence
\verb/\int/, and the \emph{limits of integration}
(in this case $a$ and $b$ are treated as a
subscript and a superscript on the integral sign.

Most integrals occurring in mathematical documents begin with
an integral sign and contain one or more instances of~$d$
followed by another (Latin or Greek) letter, as in 
$dx$, $dy$ and $dt$.
To obtain the correct appearance one should put
extra space before the~$d$,
using \verb/\,/. Thus
\[ \int_0^{+\infty} x^n e^{-x} \,dx = n!. \]
\[ \int \cos \theta \,d\theta = \sin \theta. \]
\[ \int_{x^2 + y^2 \leq R^2} f(x,y)\,dx\,dy
   = \int_{\theta=0}^{2\pi} \int_{r=0}^R
      f(r\cos\theta,r\sin\theta) r\,dr\,d\theta. \]
and
\[ \int_0^R \frac{2x\,dx}{1+x^2} = \log(1+R^2). \]
are obtained by typing
\begin{quote}
\begin{verbatim}
\[ \int_0^{+\infty} x^n e^{-x} \,dx = n!.\] 
\end{verbatim}
\end{quote}
\begin{quote}
\begin{verbatim}
\[ \int \cos \theta \,d\theta = \sin \theta.\] 
\end{verbatim}
\end{quote}
\begin{quote}
\begin{verbatim}
\[ \int_{x^2 + y^2 \leq R^2} f(x,y)\,dx\,dy
   = \int_{\theta=0}^{2\pi} \int_{r=0}^R
      f(r\cos\theta,r\sin\theta) r\,dr\,d\theta.\] 
\end{verbatim}
\end{quote}
and
\begin{quote}
\begin{verbatim}
\[ \int_0^R \frac{2x\,dx}{1+x^2} = \log(1+R^2).\] 
\end{verbatim}
\end{quote}
respectively.

In some multiple integrals (i.e., integrals containing more than
one integral sign) one finds that \LaTeX\ puts too much space
between the integral signs. The way to improve the appearance of
of the integral is to use the control sequence \verb/\!/ to
remove a thin strip of unwanted space. Thus, for example, the
multiple integral
\[ \int_0^1 \! \int_0^1 x^2 y^2\,dx\,dy. \]
is obtained by typing
\begin{quote}
\begin{verbatim}
\[ \int_0^1 \! \int_0^1 x^2 y^2\,dx\,dy.\] 
\end{verbatim}
\end{quote}
Had we typed
\begin{quote}
\begin{verbatim}
\[ \int_0^1 \int_0^1 x^2 y^2\,dx\,dy.\] 
\end{verbatim}
\end{quote}
we would have obtained
\[ \int_0^1 \int_0^1 x^2 y^2\,dx\,dy. \]

A particularly noteworthy example comes when we are
typesetting a multiple integral such as
\[ \int \!\!\! \int_D f(x,y)\,dx\,dy. \]
Here we use \verb/\!/ three times to obtain suitable spacing
between the integral signs. We typeset this integral using
\begin{quote}
\begin{verbatim}
\[ \int \!\!\! \int_D f(x,y)\,dx\,dy.\] 
\end{verbatim}
\end{quote}
Had we typed
\begin{quote}
\begin{verbatim}
\[ \int \int_D f(x,y)\,dx\,dy.\] 
\end{verbatim}
\end{quote}
we would have obtained
\[ \int \int_D f(x,y)\,dx\,dy. \]

The following (reasonably complicated) passage exhibits a
number of the features which we have been discussing:
\begin{quotation}
In non-relativistic wave mechanics, the wave function
$\psi(\mathbf{r},t)$ of a particle satisfies the
\textit{Schr\"{o}dinger Wave Equation}
\[ i\hbar\frac{\partial \psi}{\partial t}
  = \frac{-\hbar^2}{2m} \left(
    \frac{\partial^2}{\partial x^2}
    + \frac{\partial^2}{\partial y^2}
    + \frac{\partial^2}{\partial z^2}
  \right) \psi + V \psi.\] 
It is customary to normalize the wave equation by
demanding that
\[ \int \!\!\! \int \!\!\! \int_{\textbf{R}^3}
      \left| \psi(\mathbf{r},0) \right|^2\,dx\,dy\,dz = 1.\] 
A simple calculation using the Schr\"{o}dinger wave
equation shows that
\[ \frac{d}{dt} \int \!\!\! \int \!\!\! \int_{\textbf{R}^3}
      \left| \psi(\mathbf{r},t) \right|^2\,dx\,dy\,dz = 0,\] 
and hence
\[ \int \!\!\! \int \!\!\! \int_{\textbf{R}^3}
      \left| \psi(\mathbf{r},t) \right|^2\,dx\,dy\,dz = 1\] 
for all times~$t$. If we normalize the wave function in this
way then, for any (measurable) subset~$V$ of $\textbf{R}^3$
and time~$t$,
\[ \int \!\!\! \int \!\!\! \int_V
      \left| \psi(\mathbf{r},t) \right|^2\,dx\,dy\,dz\] 
represents the probability that the particle is to be found
within the region~$V$ at time~$t$.
\end{quotation}
One would typeset this in \LaTeX\ by typing
\begin{quote}
\begin{verbatim}
In non-relativistic wave mechanics, the wave function
$\psi(\mathbf{r},t)$ of a particle satisfies the
\textit{Schr\"{o}dinger Wave Equation}
\[ i\hbar\frac{\partial \psi}{\partial t}
  = \frac{-\hbar^2}{2m} \left(
    \frac{\partial^2}{\partial x^2}
    + \frac{\partial^2}{\partial y^2}
    + \frac{\partial^2}{\partial z^2}
  \right) \psi + V \psi.\] 
It is customary to normalize the wave equation by
demanding that
\[ \int \!\!\! \int \!\!\! \int_{\textbf{R}^3}
      \left| \psi(\mathbf{r},0) \right|^2\,dx\,dy\,dz = 1.\] 
A simple calculation using the Schr\"{o}dinger wave
equation shows that
\[ \frac{d}{dt} \int \!\!\! \int \!\!\! \int_{\textbf{R}^3}
      \left| \psi(\mathbf{r},t) \right|^2\,dx\,dy\,dz = 0,\] 
and hence
\[ \int \!\!\! \int \!\!\! \int_{\textbf{R}^3}
      \left| \psi(\mathbf{r},t) \right|^2\,dx\,dy\,dz = 1\] 
for all times~$t$. If we normalize the wave function in this
way then, for any (measurable) subset~$V$ of $\textbf{R}^3$
and time~$t$,
\[ \int \!\!\! \int \!\!\! \int_V
      \left| \psi(\mathbf{r},t) \right|^2\,dx\,dy\,dz\] 
represents the probability that the particle is to be found
within the region~$V$ at time~$t$.
\end{verbatim}
\end{quote}

\section{Further Features of \LaTeX}

\subsection{Producing White Space in \LaTeX}

To produce (horizontal) blank space within a paragraph, use
\verb/\hspace/, followed by the length
of the blank space enclosed within curly brackets.  The length
of the skip should be expressed in a unit recognized by \LaTeX.
These recognized units are given in the following table:
\begin{quote}
\begin{tabular}{lll}
\texttt{pt}  & point         & (1 in = 72.27 pt) \\
\texttt{pc}  & pica          & (1 pc = 12 pt) \\
\texttt{in}  & inch          & (1 in = 25.4 mm) \\
\texttt{bp}  & big point     & (1 in = 72 bp) \\
\texttt{cm}  & centimetre    & (1 cm = 10 mm) \\
\texttt{mm}  & millimetre    & \\
\texttt{dd}  & didot point   & (1157 dd = 1238 pt) \\
\texttt{cc}  & cicero        & (1 cc = 12 dd) \\
\texttt{sp}  & scaled point  & (65536 sp = 1 pt) \\
\end{tabular}
\end{quote}
Thus to produce a horizontal blank space of 20 mm in the middle
of a paragraph one would type \verb/\hspace{20 mm}/.

If \LaTeX\ decides to break between lines at a point in the document
where an \verb/\hspace/ is specified, then no white
space is produced.  To ensure that white space is produced
even at points in the document where line breaking takes place, one
should replace \verb/\hspace/ by \verb/\hspace*/

To produce (vertical) blank space between paragraphs, use
\verb/\vspace/, followed by the length of the blank space
enclosed within curly brackets.  Thus to obtain
\begin{quotation}
This is the first paragraph of some text.  It is
separated from the second paragraph by a vertical skip of
10 millimetres.

\vspace{10 mm}

This is the second paragraph.
\end{quotation}
one should type
\begin{quote}
\begin{verbatim}
This is the first paragraph of some text.  It is
separated from the second paragraph by a vertical skip of
10 millimetres.

\vspace{10 mm}
This is the second paragraph.

\end{verbatim}
\end{quote}
If \LaTeX\ decides to introduce at a point in the document
where a \verb/\vspace/ is specified, then no white
space is produced.  To ensure that white space is produced
even at points in the document where page breaking takes place, one
should replace \verb/\vspace/ by \verb/\vspace*/

We now describe certain features of \LaTeX\ relating to blank spaces
and paragraph indentation which will improve the appearance
of the final document.  Experienced users of \LaTeX\ will improve
the appearance of their documents if they bear these remarks in mind.

First note that, as a general rule, you should never put
a blank space after a left parenthesis or before a right
parenthesis.  If you were to put a blank space in these
places, then you run the risk that \LaTeX\ might start a
new line immediately after the left parenthesis or before
the right parenthesis, leaving the parenthesis marooned at
the beginning or end of a line.

\LaTeX\ has its own rules for deciding the lengths of blank
spaces.  For instance, \LaTeX\ will put an extra amount of space
after a full stop if it considers that the full stop marks the
end of a sentence.

The rule adopted by \LaTeX\ is to regard a period (full stop) as
the end of a sentence if it is preceded by a lowercase letter.
If the period is preceded by an uppercase letter then
\LaTeX\ assumes that it is not a full stop but follows the
initials of somebody's name.

This works very well in most cases.  However
\LaTeX\ occasionally gets things wrong.  This happens with
a number of common abbreviations (as in `Mr.\ Smith' or
in `etc.'), and, in particular, in the names of
journals given in abbreviated form (e.g.,
`Proc.\ Amer.\ Math.\ Soc.').  The way to overcome this
problem is to put a backslash before the blank space in
question.  Thus we should type
\begin{quote}
\begin{verbatim}
Mr.\ Smith
etc.\ and
Proc.\ Amer.\ Math.\ Soc.
\end{verbatim}
\end{quote}

\LaTeX\ determines itself how to break up a paragraph into
lines, and will occasionally hyphenate long words where this
is desirable. However it is sometimes necessary to tell
\LaTeX\ not to break at a particular blank space. The special
character used for this purpose is \verb/~/. It represents
a blank space at which \LaTeX\ is not allowed to break between
lines. It is often desirable to use \verb/~/ in names where
the forenames are represented by initials. Thus to obtain
`W. R. Hamilton' it is best to type \verb/W.~R.~Hamilton/.
It is also desirable in phrases like `Example 7' and
`the length~$l$ of the rod', obtained by typing
\verb/Example~7/ and \verb/the length~$l$ of the rod./

\LaTeX\ will automatically indent paragraphs (with the
exception of the first paragraph of a new section).  One
can prevent \LaTeX\ from indenting a paragraph though by
beginning the paragraph with the control sequence
\verb/\noindent/.  Thus one obtains
\begin{quotation}
\noindent
This is the beginning of a paragraph which is not
indented in the usual way.  This has been achieved
by placing an appropriate control sequence at the
beginning of the paragraph.
\end{quotation}
by typing
\begin{quote}
\begin{verbatim}
\noindent
This is the beginning of a paragraph which is not
indented in the usual way.  This has been achieved
by placing an appropriate control sequence at the
beginning of the paragraph.
\end{verbatim}
\end{quote}

Conversely, the control sequence \verb/\indent/ forces
\LaTeX\ to indent the paragraph.

\subsection{Lists}

\LaTeX\ provides the following list environments:
\begin{itemize}
\item \verb/enumerate/ for numbered lists,
\item \verb/itemize/ for un-numbered lists,
\item \verb/description/ for description lists
\end{itemize}

Numbered lists are produced using
\begin{quote}
\begin{verbatim}
\begin{enumerate} ... \end{enumerate}
\end{verbatim}
\end{quote}
The items in the list should be enclosed between
\begin{quote}
\verb/\begin{enumerate}/ and \verb/\end{enumerate}/
\end{quote}
and should each be preceded by the control sequence \verb/\item/
(which will automatically generate the number labelling the item).
For example, the text
\begin{quotation}
A \emph{metric space} $(X,d)$ consists of a set~$X$ on
which is defined a \emph{distance function} which assigns
to each pair of points of $X$ a distance between them,
and which satisfies the following four axioms:
\begin{enumerate}
\item
$d(x,y) \geq 0$ for all points $x$ and $y$ of $X$;
\item
$d(x,y) = d(y,x)$ for all points $x$ and $y$ of $X$;
\item
$d(x,z) \leq d(x,y) + d(y,z)$ for all points $x$, $y$
and $z$ of $X$;
\item
$d(x,y) = 0$ if and only if the points $x$ and $y$
coincide.
\end{enumerate}
\end{quotation}
is generated by \LaTeX\ from the following input:
\begin{quote}
\begin{verbatim}
A \emph{metric space} $(X,d)$ consists of a set~$X$ on
which is defined a \emph{distance function} which assigns
to each pair of points of $X$ a distance between them,
and which satisfies the following four axioms:
\begin{enumerate}
\item
$d(x,y) \geq 0$ for all points $x$ and $y$ of $X$;
\item
$d(x,y) = d(y,x)$ for all points $x$ and $y$ of $X$;
\item
$d(x,z) \leq d(x,y) + d(y,z)$ for all points $x$, $y$
and $z$ of $X$;
\item
$d(x,y) = 0$ if and only if the points $x$ and $y$
coincide.
\end{enumerate}
\end{verbatim}
\end{quote}

Un-numbered lists are produced using
\begin{quote}
\begin{verbatim}
\begin{itemize} ... \end{itemize}
\end{verbatim}
\end{quote}
If we replace
\begin{quote}
\verb/\begin{enumerate}/ and \verb/\end{enumerate}/
\end{quote}
in the above input by
\begin{quote}
\verb/\begin{itemize}/ and \verb/\end{itemize}/
\end{quote}
respectively, \LaTeX\ generates an itemized list in which each
item is preceeded by a `bullet':
\begin{quotation}
A \emph{metric space} $(X,d)$ consists of a set~$X$ on
which is defined a \emph{distance function} which assigns
to each pair of points of $X$ a distance between them,
and which satisfies the following four axioms:
\begin{itemize}
\item
$d(x,y) \geq 0$ for all points $x$ and $y$ of $X$;
\item
$d(x,y) = d(y,x)$ for all points $x$ and $y$ of $X$;
\item
$d(x,z) \leq d(x,y) + d(y,z)$ for all points $x$, $y$
and $z$ of $X$;
\item
$d(x,y) = 0$ if and only if the points $x$ and $y$
coincide.
\end{itemize}
\end{quotation}

Description lists (for glossaries etc.) are produced using
\begin{quote}
\begin{verbatim}
\begin{description} ... \end{description}
\end{verbatim}
\end{quote}
The items in the list should be enclosed between
\begin{quote}
\verb/\begin{description}/ and \verb/\end{description}/
\end{quote}
and should each be preceded by
\verb/\item[/\emph{label}\verb/]/,
where \emph{label} is the label to be assigned to each item.
For example, the text
\begin{quotation}
We now list the definitions of \emph{open ball},
\emph{open set} and \emph{closed set} in a metric space.
\begin{description}
\item[open ball]
The \emph{open ball} of radius~$r$ about any point~$x$
is the set of all points of the metric space whose
distance from $x$ is strictly less than $r$;
\item[open set]
A subset of a metric space is an \emph{open set} if,
given any point of the set, some open ball of
sufficiently small radius about that point is contained
wholly within the set;
\item[closed set]
A subset of a metric space is a \emph{closed set} if its
complement is an open set.
\end{description}
\end{quotation}
is generated by \LaTeX\ from the following input:
\begin{quote}
\begin{verbatim}
We now list the definitions of \emph{open ball},
\emph{open set} and \emph{closed set} in a metric space.
\begin{description}
\item[open ball]
The \emph{open ball} of radius~$r$ about any point~$x$
is the set of all points of the metric space whose
distance from $x$ is strictly less than $r$;
\item[open set]
A subset of a metric space is an \emph{open set} if,
given any point of the set, some open ball of
sufficiently small radius about that point is contained
wholly within the set;
\item[closed set]
A subset of a metric space is a \emph{closed set} if its
complement is an open set.
\end{description}
\end{verbatim}
\end{quote}

\subsection{Displayed Quotations}

Displayed quotations can be embedded in text using the
\textbf{quote} and \textbf{quotation} environments
\begin{quote}
\begin{verbatim}
\begin{quote} ... \end{quote}
\end{verbatim}
\end{quote}
and
\begin{quote}
\begin{verbatim}
\begin{quotation} ... \end{quotation}.
\end{verbatim}
\end{quote}
The \textbf{quote} environment is recommended for short quotations:
the whole quotation is indended in the \textbf{quote} environment,
but the first lines of individual paragraphs are not further indented.
The input file
\begin{quote}
\begin{verbatim}
Isaac Newton discovered the basic techiques of
the differential and integral calculus, and
applied them in the study of many problems
in mathematical physics.  His main mathematical
works are the \emph{Principia} and the \emph{Optics}.
He summed up his own estimate of his work as follows:
\begin{quote}
I do not know what I may appear to the world; but to
myself I seem to have been only like a boy, playing
on the sea-shore, and diverting myself, in now and
then finding a smoother pebble, or a prettier shell
than ordinary, whilst the great ocean of truth lay
all undiscovered before me.
\end{quote}
In later years Newton became embroiled in a bitter
priority dispute with Leibniz over the discovery
of the basic techniques of calculus.
\end{verbatim}
\end{quote}
is typeset by \LaTeX\ as follows:
\begin{quotation}
Isaac Newton discovered the basic techiques of
the differential and integral calculus, and
applied them in the study of many problems
in mathematical physics.  His main mathematical
works are the \emph{Principia} and the \emph{Optics}.
He summed up his own estimate of his work as follows:
\begin{quote}
I do not know what I may appear to the world; but to
myself I seem to have been only like a boy, playing
on the sea-shore, and diverting myself, in now and
then finding a smoother pebble, or a prettier shell
than ordinary, whilst the great ocean of truth lay
all undiscovered before me.
\end{quote}
In later years Newton became embroiled in a bitter
priority dispute with Leibniz over the discovery
of the basic techniques of calculus.
\end{quotation}

For longer quotations one may use the \textbf{quotation}
environment: the whole quotation is indented, and the openings
of paragraphs are then further indented in the normal fashion.

\subsection{Tables}

Tables can be produced in \LaTeX\ using the \textbf{tabular} environment.
For example, the text
\begin{quotation}
The first five International Congresses of Mathematicians
were held in the following cities:
\begin{quote}
\begin{tabular}{lll}
Chicago&U.S.A.&1893\\
Z\"{u}rich&Switzerland&1897\\
Paris&France&1900\\
Heidelberg&Germany&1904\\
Rome&Italy&1908
\end{tabular}
\end{quote}
\end{quotation}
is produced in \LaTeX using the following input file:
\begin{quote}
\begin{verbatim}
The first five International Congresses of Mathematicians
were held in the following cities:
\begin{quote}
\begin{tabular}{lll}
Chicago&U.S.A.&1893\\
Z\"{u}rich&Switzerland&1897\\
Paris&France&1900\\
Heidelberg&Germany&1904\\
Rome&Italy&1908
\end{tabular}
\end{quote}
\end{verbatim}
\end{quote}
The \verb/\begin{tabular}/ command must be followed by a string
of characters enclosed within braces which specifies the format
of the table.  In the above example, the string \verb/{lll}/ is a
format specification for a table with three columns of left-justified text.
Within the body of the table the ampersand character~\verb/&/ is used
to separate columns of text within each row, and the double
backslash~\verb/\\/ is used to separate the rows of the table.

The next example shows how to obtain a table with
vertical and horizontal lines.  The text
\begin{quotation}
The group of permutations of a set of $n$~elements has
order $n!$, where $n!$, the factorial of $n$, is the
product of all integers between $1$ and $n$.  The
following table lists the values of the factorial of each
integer~$n$ between 1 and 10:
\begin{quote}
\begin{tabular}{|r|r|}
\hline
$n$&$n!$\\
\hline
1&1\\
2&2\\
3&6\\
4&24\\
5&120\\
6&720\\
7&5040\\
8&40320\\
9&362880\\
10&3628800\\
\hline
\end{tabular}
\end{quote}
Note how rapidly the value of $n!$ increases with $n$.
\end{quotation}
is produced in \LaTeX using the following input file:
\begin{quote}
\begin{verbatim}
The group of permutations of a set of $n$~elements has
order $n!$, where $n!$, the factorial of $n$, is the
product of all integers between $1$ and $n$.  The
following table lists the values of the factorial of each
integer~$n$ between 1 and 10:
\begin{quote}
\begin{tabular}{|r|r|}
\hline
$n$&$n!$\\
\hline
1&1\\
2&2\\
3&6\\
4&24\\
5&120\\
6&720\\
7&5040\\
8&40320\\
9&362880\\
10&3628800\\
\hline
\end{tabular}
\end{quote}
Note how rapidly the value of $n!$ increases with $n$.
\end{verbatim}
\end{quote}
In this example the format specification \verb/{|r|r|}/ after
\verb/\begin{tabular}/ specifies that the table should consist
of two columns of right-justified text, with vertical lines
to the left and to the right of the table, and between columns.

Within the body of the table, the command \verb/\hline/ produces
a horizontal line; this command can only be placed between the
format specification and the body of the table (to produce a line
along the top of the table) or immediately after a row separator
(to produce a horizontal line between rows or at the bottom of the
table).

In a \textbf{tabular} environment, the format specification after
\verb/\begin{tabular}/ should consist of one or more of the following,
enclosed within braces \verb/{/ and \verb/}/:
\begin{quote}
\begin{tabular}{ll}
\verb/l/&specifies a column of left-justified text\\
\verb/c/&specifies a column of centred text\\
\verb/r/&specifies a column of right-justified text\\
\verb/p{/\emph{width}\verb/}/&specifies a left-justified column
   of the given width\\
\verb/|/&inserts a vertical line between columns\\
\verb/@{/\emph{text}\verb/}/&inserts the given \emph{text}
   between columns\\
\end{tabular}
\end{quote}

A string \emph{str} of characters in the format specification can be
repeated \emph{num} times using the construction
\verb/*{/\emph{num}\verb/}{/\emph{str}\verb/}/.  For example,
a table with 15 columns of right-justified text enclosed within
vertical lines can be produced using the format specification
\verb/{|*{15}{r|}}/.

If additional vertical space is required between rows of the table,
then this can be produced by specifying the amount of space within
square brackets after \verb/\\/.  For example, on would use
\verb/\\[6pt]/ to separate two rows of the table by 6~points of
blank space.

A horizontal line in a table from column~$i$ to column~$j$ inclusive
can be produced using \verb/\cline{/$i$\verb/-/$j$\verb/}/.  For
example \verb/\cline{3-5}/ produces a horizontal line spanning
columns 3, 4 and 5 of some table.

A command of the form
\verb/\multicolumn{/\emph{num}\verb/}{/\emph{fmt}\verb/}{/\emph{text}\verb/}/
can be used within the body of a table to produce an entry spanning
several columns.  Here \emph{num} specifies the number of columns
to be spanned, \emph{fmt} specifies the format for the entry
(e.g., \verb/l/ if the entry is to be left-justified entry,
or \verb/c/ if the entry is to be centred), and \emph{text} is
the text of the entry.  For example, to span three columns
of a table with the words `Year of Entry' (centred with respect
to the three columns), one would use
\begin{quote}
\begin{verbatim}
\multicolumn{3}{c}{Year of entry}
\end{verbatim}
\end{quote}

\subsection{The Preamble of the \LaTeX\ Input file}

We describe the options available in \LaTeX\ for specifying
the overall style of a document.

A \LaTeX\ document should begin with a \verb/\documentclass/
command and any text to be printed must be included between
\begin{quote}
\verb/\begin{document}/ and \verb/\end{document}/
\end{quote}
commands.  The \verb/\begin{document}/ command is sometimes
preceded by commands that set the page-style and set up
user-defined control sequences.

Here is a typical \LaTeX\ input file:
\begin{quote}
\begin{verbatim}
\documentclass[a4paper,12pt]{article}
\begin{document}

This is the first paragraph of a typical document. It is
produced in a `12~point' size. A \textit{point} is a unit
of length used by printers. One point is approximately
$1/72$~inch. In a `12~point' font the height of the
parentheses is 12~points (i.e. about $1/6$~inch) and the
letter~`m' is about 12 points long. 

This is the second paragraph of the document. There are
also `10 point' and `11 point' styles available in \LaTeX.
The required size is specified in the `documentstyle'
command. If no such size is specified then the 10~point
size is assumed.

\end{document}
\end{verbatim}
\end{quote}

The syntax of the \verb/\documentclass/ command is as
follows. The command begins with \verb/\documentclass/
and ends with the names of one of the available styles,
enclosed in curly brackets. The available styles are
\verb/article/, \verb/report/, \verb/book/
and \verb/letter/. Between the ``\verb/\documentclass/''
and the name of the document style, one may place a
list of \emph{options}. These options are separated by
commas and the list of options is enclosed in square
brackets (as in the above example). The options
available (which are usually the names of certain
`style files') include the following:
\begin{description}
\item[11pt]
   Specifies a size of type known as
   \emph{eleven-point}, which is ten percent larger than
   the ten-point type normally used.
\item[12pt]
   Specifies a twelve-point type size, which is
   twenty percent larger than ten-point.
\item[twocolumn]
   Produces two-column output.
\item[a4paper]
   This ensures that the page is appropriately positioned
   on A4 size paper.
\end{description}

Typing simply \verb/\documentclass{article}/ will produce a
document in ten-point type size. However the printed output
will not be nicely positioned on A4 paper, since the default
size is intended for a different (American) paper size.

Pages will be automatically numbered at the bottom of the
page, unless you specify otherwise. This can be done using the
\verb/\pagestyle/ command. This command should come after the
\verb/\documentclass/ command and before the
\verb/\begin{document}/ command. This command has the
syntax \verb/\pagestyle{/\textit{option}\verb/}/, where
the \textit{option} is one of the following:
\begin{description}
\item[plain]
   The page number is at the foot of the page.
   This is the default page style for the \verb/article/
   and \verb/report/ document styles.
\item[empty]
   No page number is printed.
\item[headings]
   The page number (and any other information
   determined by the document style) is put at the top of
   the page.
\item[myheadings]
   Similar to the \textbf{headings} pagestyle,
   except that the material to go at the top of the
   page is determined by \verb/\markboth/ and
   \verb/\markright/ commands (see the \LaTeX\ manual).
\end{description}
For example, the input file
\begin{quote}
\begin{verbatim}
\documentclass[a4paper]{article}
\pagestyle{empty}
\begin{document}
The main body of the document is placed here.
\end{document}
\end{verbatim}
\end{quote}
produces a document without page numbers, using the
standard ten-point type size.

\subsection{Defining your own Control Sequences in \LaTeX}

Suppose that we are producing a paper that makes frequent
use of some mathematical expression.  For example,
suppose that integrals like
\[ \int_{-\infty}^{+\infty} f(x)\,dx. \]
occur frequently throughout the text.  This formula is
obtained by typing
\begin{quote}
\begin{verbatim}
\[ \int_{-\infty}^{+\infty} f(x)\,dx.\] 
\end{verbatim}
\end{quote}
It would be nice if we could type \verb/\inftyint/ (say)
to obtain the integral sign at the beginning.  This can
be done using \verb/\newcommand/.  What we do is to place
a line with the command
\begin{quote}
\begin{verbatim}
\newcommand{\inftyint}{\int_{-\infty}^{+\infty}}
\end{verbatim}
\end{quote}
near the beginning of the input file (e.g., after
the \verb/\documentclass/ command but before the
\verb/\begin{document}/ command).  Then we only have to
type
\begin{quote}
\begin{verbatim}
\[ \inftyint f(x)\,dx.\] 
\end{verbatim}
\end{quote}
to obtain the above formula.

We can modify this procedure slightly.  Suppose that we
we defined a new control sequence \verb/\intwrtx/ by
putting the line
\begin{quote}
\begin{verbatim}
\newcommand{\intwrtx}[1]{\int_{-\infty}^{+\infty} #1 \,dx}
\end{verbatim}
\end{quote}
at the beginning of the input file.  If we then type the line
\begin{quote}
\begin{verbatim}
\[ \intwrtx{f(x)}.\] 
\end{verbatim}
\end{quote}
then we obtain
\newcommand{\intwrtx}[1]{\int_{-\infty}^{+\infty} #1 \,dx}
\[ \intwrtx{f(x)}. \]
What has happened is that the expression in curly brackets
after \verb/\intwrtx/ has been substituted in the expression
defining \verb/\intwrtx/, replacing the \verb/#1/ in that
expression.

The number 1 inside square brackets in the
\verb/\newcommand/ line defining \verb/\intwrtx/ indicates
to \LaTeX\ that it is to expect one expression (in curly
brackets) after \verb/\intwrtx/ to substitute for \verb/#1/
in the definition of \verb/\intwrtx/.  If we defined a
control sequence \verb/\intwrt/ by
\begin{quote}
\begin{verbatim}
\newcommand{\intwrt}[2]{\int_{-\infty}^{+\infty} #2 \,d #1}
\end{verbatim}
\end{quote}
then it would expect two expressions to substitute in for
\verb/#1/ and \verb/#2/ in the  definition of \verb/\intwrt/.
Thus  if we then type
\begin{quote}
\begin{verbatim}
\[ \intwrt{y}{f(y)}.\] 
\end{verbatim}
\end{quote}
we obtain
\newcommand{\intwrt}[2]{\int_{-\infty}^{+\infty} #2 \,d #1}
\[ \intwrt{y}{f(y)}. \]

\end{document}
