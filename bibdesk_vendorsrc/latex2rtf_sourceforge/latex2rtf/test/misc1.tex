%--------------------------------------------------------------------
%  Basisfile um alle Kapitel auf einmal laufen zu lassen
%--------------------------------------------------------------------

\documentclass[12pt , twoside , german ,a4paper]{report}
\usepackage{verbatim}
\usepackage{a4}

\usepackage{babel}
\usepackage{colortbl,array,tabularx,graphics,color} %f\"ur besondere Tabellengestaltung
\usepackage{fancyhdr} %f�r Seitenkopf und -fu�
\usepackage{latexsym} %f�r Symbole (z.B. Box bei Beweisende)
\usepackage{epic,eepic,ecltree,epsfig}

\setlength{\parindent}{0pt}	%kein Einruecken am Absatzanfang
\setlength{\parskip}{0.75\baselineskip}	
\setlength{\baselineskip}{4.2ex}				  	

\setlength{\textheight}{23cm}     
\setlength{\evensidemargin}{0.5cm}
\setlength{\oddsidemargin}{0.5cm}
\setlength{\headsep}{1cm}  

\renewcommand{\textfraction}{0.1}
\renewcommand{\topfraction}{0.9}
\renewcommand{\bottomfraction}{0.9}

%-----------------   Definitionen + Abkuerzungen   ------------------
\def\N{I\!\!N}
\def\R{I\!\!R}
\newcommand{\fks}{Fall-Kontroll-Studie}
\newcommand{\fksn}{Fall-Kontroll-Studien}
\newcommand{\rnl}{Rhein-Neckar-Larynx Studie}
\newcommand{\tsd}{Two-Stage-Design}
\newcommand{\tsds}{Two-Stage-Designs}
\newcommand{\hd}{h-Design}

%-----------------   Kopfzeile  --------------------------------------
\newcommand{\kzl}{~}	%--K\"urzel f\"ur Kopfzeile --
\newcommand{\kzr}{~}
\renewcommand{\headrulewidth}{0.4pt}
%-----------------   Textanfang --------------------------------------

\begin{document}

\hd

\tsd

R=(\(R_{1},..,R_{r}) \)

\"Ose

l\"a\ss{}t
\"Anderung \"Ubung

Nach Little \& Rubin (1987) kann die Response-Wahrscheinlichkeit abh\"angen\\
\begin{tabular}{lll}
~~~~& (1) von D und m\"oglicherweise auch R, 	& (weder MAR noch OAR)\\
		& (2) von R, aber nicht von D, 					& (MAR, aber nicht OAR)\\
		& (3) weder von R noch von D 						& (MAR und OAR)\\
\end{tabular}

\begin{description}
\item[\textbf{Definition 4.1:}]             \mbox{}\\
Sei D eine bin\"are Responsevariable und seien $R_{1}$ und $R_{2}$ zwei beliebige (d.h. kategorielle oder stetige) unabh\"angige Variablen oder auch Variablen-Vektoren. O.B.d.A. sei die Kovariable $R_{1}$ f\"ur alle Beobachtungen vollst\"andig, die Kovariable $R_{2}$ nicht vollst\"andig vorhanden.
\renewcommand{\theenumi}{(\roman{enumi})}
\renewcommand{\labelenumi}{\theenumi.}
\begin{enumerate}
\item Die Beobachtbarkeit $O_{2}$ von $R_{2}$ ist definiert als Zufallsvariable
\begin{displaymath}
O_{2}:= \left\{
\begin{array}{ll}
1&~~~\mbox{ wenn $R_{2}$ beobachtbar ist,} \\
0&~~~\mbox{ wenn $R_{2}$ nicht beobachtbar ist.}
\end{array}
\right.
\end{displaymath}	
\item Fehlende Daten sind genau dann {\it missing at random}, wenn f\"ur i = 0,1 gilt:\\ 
\(P(O_{2}=1 \mid D=i, R_{1}=r_{1}, R_{2}=r_{2}) \equiv  P ( O_{2}=1 \mid D=i, R_{1}=r_{1})\)\\
\item Fehlende Daten sind genau dann {\it missing completely at random}, wenn f\"ur i = 0,1 gilt:\\ 
\(P(O_{2}=1 \mid D=i, R_{1}=r_{1}, R_{2}=r_{2}) \equiv  P ( O_{2}=1 ) \)         
\end{enumerate}
\end{description}


\end{document}



